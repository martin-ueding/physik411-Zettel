% Copyright © 2013 Martin Ueding <dev@martin-ueding.de>
%
% Copyright © 2012-2013 Martin Ueding <dev@martin-ueding.de>

% This is my general purpose LaTeX header file for writing German documents.
% Ideally, you include this using a simple ``% Copyright © 2012-2013 Martin Ueding <dev@martin-ueding.de>

% This is my general purpose LaTeX header file for writing German documents.
% Ideally, you include this using a simple ``% Copyright © 2012-2013 Martin Ueding <dev@martin-ueding.de>

% This is my general purpose LaTeX header file for writing German documents.
% Ideally, you include this using a simple ``\input{header.tex}`` in your main
% document and start with ``\title`` and ``\begin{document}`` afterwards.

% If you need to add additional packages, I recommend not doing this in this
% file, but in your main document. That way, you can just drop in a new
% ``header.tex`` and get all the new commands without having to merge manually.

% Since this file encorporates a CC-BY-SA fragment, this whole files is
% licensed under the CC-BY-SA license.

\documentclass[11pt, ngerman, fleqn, DIV=15, headinclude]{scrartcl}

\usepackage{graphicx}

% Environment to quote the problem. Currently, this is just a new name for the
% quote environment.
\newenvironment{problem}{\begin{quote}}{\end{quote}}

%%%%%%%%%%%%%%%%%%%%%%%%%%%%%%%%%%%%%%%%%%%%%%%%%%%%%%%%%%%%%%%%%%%%%%%%%%%%%%%
%                                Locale, date                                 %
%%%%%%%%%%%%%%%%%%%%%%%%%%%%%%%%%%%%%%%%%%%%%%%%%%%%%%%%%%%%%%%%%%%%%%%%%%%%%%%

\usepackage{babel}
\usepackage[iso]{isodate}

%%%%%%%%%%%%%%%%%%%%%%%%%%%%%%%%%%%%%%%%%%%%%%%%%%%%%%%%%%%%%%%%%%%%%%%%%%%%%%%
%                          Margins and other spacing                          %
%%%%%%%%%%%%%%%%%%%%%%%%%%%%%%%%%%%%%%%%%%%%%%%%%%%%%%%%%%%%%%%%%%%%%%%%%%%%%%%

\usepackage[parfill]{parskip}
\usepackage{setspace}
\usepackage[activate]{microtype}

\setlength{\columnsep}{2cm}

%%%%%%%%%%%%%%%%%%%%%%%%%%%%%%%%%%%%%%%%%%%%%%%%%%%%%%%%%%%%%%%%%%%%%%%%%%%%%%%
%                                    Color                                    %
%%%%%%%%%%%%%%%%%%%%%%%%%%%%%%%%%%%%%%%%%%%%%%%%%%%%%%%%%%%%%%%%%%%%%%%%%%%%%%%

\usepackage[usenames, dvipsnames]{xcolor}

\colorlet{darkred}{red!70!black}
\colorlet{darkblue}{blue!70!black}
\colorlet{darkgreen}{green!40!black}

%%%%%%%%%%%%%%%%%%%%%%%%%%%%%%%%%%%%%%%%%%%%%%%%%%%%%%%%%%%%%%%%%%%%%%%%%%%%%%%
%                         Font and font like settings                         %
%%%%%%%%%%%%%%%%%%%%%%%%%%%%%%%%%%%%%%%%%%%%%%%%%%%%%%%%%%%%%%%%%%%%%%%%%%%%%%%

% This replaces all fonts with Bitstream Charter, Bitstream Vera Sans and
% Bitstream Vera Mono. Math will be rendered in Charter.
\usepackage[charter, greekuppercase=italicized]{mathdesign}
\usepackage{beramono}
\usepackage{berasans}

% Bold, sans-serif tensors. This fragment is taken from “egreg” from
% http://tex.stackexchange.com/a/82747/8945 and licensed under `CC-BY-SA
% <https://creativecommons.org/licenses/by-sa/3.0/>`_.
\usepackage{bm}
\DeclareMathAlphabet{\mathsfit}{\encodingdefault}{\sfdefault}{m}{sl}
\SetMathAlphabet{\mathsfit}{bold}{\encodingdefault}{\sfdefault}{bx}{sl}
\newcommand{\tens}[1]{\bm{\mathsfit{#1}}}

% Bold vectors.
\renewcommand{\vec}[1]{\boldsymbol{#1}}

%%%%%%%%%%%%%%%%%%%%%%%%%%%%%%%%%%%%%%%%%%%%%%%%%%%%%%%%%%%%%%%%%%%%%%%%%%%%%%%
%                               Input encoding                                %
%%%%%%%%%%%%%%%%%%%%%%%%%%%%%%%%%%%%%%%%%%%%%%%%%%%%%%%%%%%%%%%%%%%%%%%%%%%%%%%

\usepackage[T1]{fontenc}
\usepackage[utf8]{inputenc}

%%%%%%%%%%%%%%%%%%%%%%%%%%%%%%%%%%%%%%%%%%%%%%%%%%%%%%%%%%%%%%%%%%%%%%%%%%%%%%%
%                         Hyperrefs and PDF metadata                          %
%%%%%%%%%%%%%%%%%%%%%%%%%%%%%%%%%%%%%%%%%%%%%%%%%%%%%%%%%%%%%%%%%%%%%%%%%%%%%%%

\usepackage{hyperref}
\usepackage{lastpage}

% This sets the author in the properties of the PDF as well. If you want to
% change it, just override it with another ``\hypersetup`` call.
\hypersetup{
	breaklinks=false,
	citecolor=darkgreen,
	colorlinks=true,
	linkcolor=darkblue,
	menucolor=black,
	pdfauthor={Martin Ueding},
	urlcolor=darkblue,
}

%%%%%%%%%%%%%%%%%%%%%%%%%%%%%%%%%%%%%%%%%%%%%%%%%%%%%%%%%%%%%%%%%%%%%%%%%%%%%%%
%                               Math Operators                                %
%%%%%%%%%%%%%%%%%%%%%%%%%%%%%%%%%%%%%%%%%%%%%%%%%%%%%%%%%%%%%%%%%%%%%%%%%%%%%%%

% AMS environments like ``align`` and theorems like ``proof``.
\usepackage{amsmath}
\usepackage{amsthm}

% Common math constructs like partial derivatives.
\usepackage{commath}

% Physical units.
\usepackage[output-decimal-marker={,}]{siunitx}

% Word like operators.
\DeclareMathOperator{\acosh}{arcosh}
\DeclareMathOperator{\arcosh}{arcosh}
\DeclareMathOperator{\arcsinh}{arsinh}
\DeclareMathOperator{\arsinh}{arsinh}
\DeclareMathOperator{\asinh}{arsinh}
\DeclareMathOperator{\card}{card}
\DeclareMathOperator{\csch}{cshs}
\DeclareMathOperator{\diam}{diam}
\DeclareMathOperator{\sech}{sech}
\renewcommand{\Im}{\mathop{{}\mathrm{Im}}\nolimits}
\renewcommand{\Re}{\mathop{{}\mathrm{Re}}\nolimits}

% Fourier transform.
\DeclareMathOperator{\fourier}{\ensuremath{\mathcal{F}}}

% Roman versions of “e” and “i” to serve as Euler's number and the imaginary
% constant.
\newcommand{\ee}{\eup}
\newcommand{\eup}{\mathrm e}
\newcommand{\ii}{\iup}
\newcommand{\iup}{\mathrm i}

% Symbols for the various mathematical fields (natural numbers, integers,
% rational numbers, real numbers, complex numbers).
\newcommand{\C}{\ensuremath{\mathbb C}}
\newcommand{\N}{\ensuremath{\mathbb N}}
\newcommand{\Q}{\ensuremath{\mathbb Q}}
\newcommand{\R}{\ensuremath{\mathbb R}}
\newcommand{\Z}{\ensuremath{\mathbb Z}}

% Shape like operators.
\DeclareMathOperator{\dalambert}{\Box}
\DeclareMathOperator{\laplace}{\bigtriangleup}
\newcommand{\curl}{\vnabla \times}
\newcommand{\divergence}[1]{\inner{\vnabla}{#1}}
\newcommand{\vnabla}{\vec \nabla}

\newcommand{\half}{\frac 12}

% Unit vector (German „Einheitsvektor“).
\newcommand{\ev}{\hat{\vec e}}

% Scientific notation for large numbers.
\newcommand{\e}[1]{\cdot 10^{#1}}

% Mathematician's notation for the inner (scalar, dot) product.
\newcommand{\bracket}[1]{\left\langle #1 \right\rangle}
\newcommand{\inner}[2]{\bracket{#1, #2}}

% Placeholders.
\newcommand{\emesswert}{\del{\messwert \pm \messwert}}
\newcommand{\fehlt}{\textcolor{darkred}{Hier fehlen noch Inhalte.}}
\newcommand{\messwert}{\textcolor{blue}{\square}}
\newcommand{\punkte}{\phantom{xxxxx}}
\newcommand{\punktevon}[1]{\begin{flushright}/ #1\end{flushright}}

% Separator for equations on a single line.
\newcommand{\eqnsep}{,\quad}

% Quantum Mechanics
\newcommand{\braket}[2]{\left\langle #1 \left. \vphantom{#1 #2} \right| #2 \right\rangle}
\newcommand{\braopket}[3]{\left\langle #1 \left. \vphantom{#1 #2 #3} \right| #2 \left. \vphantom{#1 #2 #3} \right| #3 \right\rangle}
\newcommand{\bra}[1]{\left\langle #1 \right|}
\newcommand{\ketbra}[2]{\left| #1 \vphantom{#2} \right\rangle \left\langle #2  \vphantom{#1} \right|}
\newcommand{\ket}[1]{\left| #1 \right\rangle}

%%%%%%%%%%%%%%%%%%%%%%%%%%%%%%%%%%%%%%%%%%%%%%%%%%%%%%%%%%%%%%%%%%%%%%%%%%%%%%%
%                                  Headings                                   %
%%%%%%%%%%%%%%%%%%%%%%%%%%%%%%%%%%%%%%%%%%%%%%%%%%%%%%%%%%%%%%%%%%%%%%%%%%%%%%%

% This will set fancy headings to the top of the page. The page number will be
% accompanied by the total number of pages. That way, you will know if any page
% is missing.
%
% If you do not want this for your document, you can just use
% ``\pagestyle{plain}``.

\usepackage{scrpage2}

\pagestyle{scrheadings}
\automark{section}
\cfoot{\footnotesize{Seite \thepage\ / \pageref{LastPage}}}
\chead{}
\ihead{}
\ohead{\rightmark}
\setheadsepline{.4pt}

%%%%%%%%%%%%%%%%%%%%%%%%%%%%%%%%%%%%%%%%%%%%%%%%%%%%%%%%%%%%%%%%%%%%%%%%%%%%%%%
%                            Bibliography (BibTeX)                            %
%%%%%%%%%%%%%%%%%%%%%%%%%%%%%%%%%%%%%%%%%%%%%%%%%%%%%%%%%%%%%%%%%%%%%%%%%%%%%%%

\newcommand{\bibliographyfile}{../../zentrale_BibTeX/Central}
\bibliographystyle{apalike2}

%%%%%%%%%%%%%%%%%%%%%%%%%%%%%%%%%%%%%%%%%%%%%%%%%%%%%%%%%%%%%%%%%%%%%%%%%%%%%%%
%                                Abbreviations                                %
%%%%%%%%%%%%%%%%%%%%%%%%%%%%%%%%%%%%%%%%%%%%%%%%%%%%%%%%%%%%%%%%%%%%%%%%%%%%%%%

\newcommand{\dhabk}{\mbox{d.\,h.}}

%%%%%%%%%%%%%%%%%%%%%%%%%%%%%%%%%%%%%%%%%%%%%%%%%%%%%%%%%%%%%%%%%%%%%%%%%%%%%%%
%                                  Licences                                   %
%%%%%%%%%%%%%%%%%%%%%%%%%%%%%%%%%%%%%%%%%%%%%%%%%%%%%%%%%%%%%%%%%%%%%%%%%%%%%%%

\usepackage{ccicons}

\newcommand{\ccbysadetext}{%
	\begin{small}
		Dieses Werk bzw. Inhalt steht unter einer
		\href{http://creativecommons.org/licenses/by-sa/3.0/deed.de}{%
			Creative Commons Namensnennung - Weitergabe unter gleichen
		Bedingungen 3.0 Unported Lizenz}.
	\end{small}
}

\newcommand{\ccbysadetitle}{%
	Lizenz: \href{http://creativecommons.org/licenses/by-sa/3.0/deed.de}
	{CC-BY-SA 3.0 \ccbysa}
}
`` in your main
% document and start with ``\title`` and ``\begin{document}`` afterwards.

% If you need to add additional packages, I recommend not doing this in this
% file, but in your main document. That way, you can just drop in a new
% ``header.tex`` and get all the new commands without having to merge manually.

% Since this file encorporates a CC-BY-SA fragment, this whole files is
% licensed under the CC-BY-SA license.

\documentclass[11pt, ngerman, fleqn, DIV=15, headinclude]{scrartcl}

\usepackage{graphicx}

% Environment to quote the problem. Currently, this is just a new name for the
% quote environment.
\newenvironment{problem}{\begin{quote}}{\end{quote}}

%%%%%%%%%%%%%%%%%%%%%%%%%%%%%%%%%%%%%%%%%%%%%%%%%%%%%%%%%%%%%%%%%%%%%%%%%%%%%%%
%                                Locale, date                                 %
%%%%%%%%%%%%%%%%%%%%%%%%%%%%%%%%%%%%%%%%%%%%%%%%%%%%%%%%%%%%%%%%%%%%%%%%%%%%%%%

\usepackage{babel}
\usepackage[iso]{isodate}

%%%%%%%%%%%%%%%%%%%%%%%%%%%%%%%%%%%%%%%%%%%%%%%%%%%%%%%%%%%%%%%%%%%%%%%%%%%%%%%
%                          Margins and other spacing                          %
%%%%%%%%%%%%%%%%%%%%%%%%%%%%%%%%%%%%%%%%%%%%%%%%%%%%%%%%%%%%%%%%%%%%%%%%%%%%%%%

\usepackage[parfill]{parskip}
\usepackage{setspace}
\usepackage[activate]{microtype}

\setlength{\columnsep}{2cm}

%%%%%%%%%%%%%%%%%%%%%%%%%%%%%%%%%%%%%%%%%%%%%%%%%%%%%%%%%%%%%%%%%%%%%%%%%%%%%%%
%                                    Color                                    %
%%%%%%%%%%%%%%%%%%%%%%%%%%%%%%%%%%%%%%%%%%%%%%%%%%%%%%%%%%%%%%%%%%%%%%%%%%%%%%%

\usepackage[usenames, dvipsnames]{xcolor}

\colorlet{darkred}{red!70!black}
\colorlet{darkblue}{blue!70!black}
\colorlet{darkgreen}{green!40!black}

%%%%%%%%%%%%%%%%%%%%%%%%%%%%%%%%%%%%%%%%%%%%%%%%%%%%%%%%%%%%%%%%%%%%%%%%%%%%%%%
%                         Font and font like settings                         %
%%%%%%%%%%%%%%%%%%%%%%%%%%%%%%%%%%%%%%%%%%%%%%%%%%%%%%%%%%%%%%%%%%%%%%%%%%%%%%%

% This replaces all fonts with Bitstream Charter, Bitstream Vera Sans and
% Bitstream Vera Mono. Math will be rendered in Charter.
\usepackage[charter, greekuppercase=italicized]{mathdesign}
\usepackage{beramono}
\usepackage{berasans}

% Bold, sans-serif tensors. This fragment is taken from “egreg” from
% http://tex.stackexchange.com/a/82747/8945 and licensed under `CC-BY-SA
% <https://creativecommons.org/licenses/by-sa/3.0/>`_.
\usepackage{bm}
\DeclareMathAlphabet{\mathsfit}{\encodingdefault}{\sfdefault}{m}{sl}
\SetMathAlphabet{\mathsfit}{bold}{\encodingdefault}{\sfdefault}{bx}{sl}
\newcommand{\tens}[1]{\bm{\mathsfit{#1}}}

% Bold vectors.
\renewcommand{\vec}[1]{\boldsymbol{#1}}

%%%%%%%%%%%%%%%%%%%%%%%%%%%%%%%%%%%%%%%%%%%%%%%%%%%%%%%%%%%%%%%%%%%%%%%%%%%%%%%
%                               Input encoding                                %
%%%%%%%%%%%%%%%%%%%%%%%%%%%%%%%%%%%%%%%%%%%%%%%%%%%%%%%%%%%%%%%%%%%%%%%%%%%%%%%

\usepackage[T1]{fontenc}
\usepackage[utf8]{inputenc}

%%%%%%%%%%%%%%%%%%%%%%%%%%%%%%%%%%%%%%%%%%%%%%%%%%%%%%%%%%%%%%%%%%%%%%%%%%%%%%%
%                         Hyperrefs and PDF metadata                          %
%%%%%%%%%%%%%%%%%%%%%%%%%%%%%%%%%%%%%%%%%%%%%%%%%%%%%%%%%%%%%%%%%%%%%%%%%%%%%%%

\usepackage{hyperref}
\usepackage{lastpage}

% This sets the author in the properties of the PDF as well. If you want to
% change it, just override it with another ``\hypersetup`` call.
\hypersetup{
	breaklinks=false,
	citecolor=darkgreen,
	colorlinks=true,
	linkcolor=darkblue,
	menucolor=black,
	pdfauthor={Martin Ueding},
	urlcolor=darkblue,
}

%%%%%%%%%%%%%%%%%%%%%%%%%%%%%%%%%%%%%%%%%%%%%%%%%%%%%%%%%%%%%%%%%%%%%%%%%%%%%%%
%                               Math Operators                                %
%%%%%%%%%%%%%%%%%%%%%%%%%%%%%%%%%%%%%%%%%%%%%%%%%%%%%%%%%%%%%%%%%%%%%%%%%%%%%%%

% AMS environments like ``align`` and theorems like ``proof``.
\usepackage{amsmath}
\usepackage{amsthm}

% Common math constructs like partial derivatives.
\usepackage{commath}

% Physical units.
\usepackage[output-decimal-marker={,}]{siunitx}

% Word like operators.
\DeclareMathOperator{\acosh}{arcosh}
\DeclareMathOperator{\arcosh}{arcosh}
\DeclareMathOperator{\arcsinh}{arsinh}
\DeclareMathOperator{\arsinh}{arsinh}
\DeclareMathOperator{\asinh}{arsinh}
\DeclareMathOperator{\card}{card}
\DeclareMathOperator{\csch}{cshs}
\DeclareMathOperator{\diam}{diam}
\DeclareMathOperator{\sech}{sech}
\renewcommand{\Im}{\mathop{{}\mathrm{Im}}\nolimits}
\renewcommand{\Re}{\mathop{{}\mathrm{Re}}\nolimits}

% Fourier transform.
\DeclareMathOperator{\fourier}{\ensuremath{\mathcal{F}}}

% Roman versions of “e” and “i” to serve as Euler's number and the imaginary
% constant.
\newcommand{\ee}{\eup}
\newcommand{\eup}{\mathrm e}
\newcommand{\ii}{\iup}
\newcommand{\iup}{\mathrm i}

% Symbols for the various mathematical fields (natural numbers, integers,
% rational numbers, real numbers, complex numbers).
\newcommand{\C}{\ensuremath{\mathbb C}}
\newcommand{\N}{\ensuremath{\mathbb N}}
\newcommand{\Q}{\ensuremath{\mathbb Q}}
\newcommand{\R}{\ensuremath{\mathbb R}}
\newcommand{\Z}{\ensuremath{\mathbb Z}}

% Shape like operators.
\DeclareMathOperator{\dalambert}{\Box}
\DeclareMathOperator{\laplace}{\bigtriangleup}
\newcommand{\curl}{\vnabla \times}
\newcommand{\divergence}[1]{\inner{\vnabla}{#1}}
\newcommand{\vnabla}{\vec \nabla}

\newcommand{\half}{\frac 12}

% Unit vector (German „Einheitsvektor“).
\newcommand{\ev}{\hat{\vec e}}

% Scientific notation for large numbers.
\newcommand{\e}[1]{\cdot 10^{#1}}

% Mathematician's notation for the inner (scalar, dot) product.
\newcommand{\bracket}[1]{\left\langle #1 \right\rangle}
\newcommand{\inner}[2]{\bracket{#1, #2}}

% Placeholders.
\newcommand{\emesswert}{\del{\messwert \pm \messwert}}
\newcommand{\fehlt}{\textcolor{darkred}{Hier fehlen noch Inhalte.}}
\newcommand{\messwert}{\textcolor{blue}{\square}}
\newcommand{\punkte}{\phantom{xxxxx}}
\newcommand{\punktevon}[1]{\begin{flushright}/ #1\end{flushright}}

% Separator for equations on a single line.
\newcommand{\eqnsep}{,\quad}

% Quantum Mechanics
\newcommand{\braket}[2]{\left\langle #1 \left. \vphantom{#1 #2} \right| #2 \right\rangle}
\newcommand{\braopket}[3]{\left\langle #1 \left. \vphantom{#1 #2 #3} \right| #2 \left. \vphantom{#1 #2 #3} \right| #3 \right\rangle}
\newcommand{\bra}[1]{\left\langle #1 \right|}
\newcommand{\ketbra}[2]{\left| #1 \vphantom{#2} \right\rangle \left\langle #2  \vphantom{#1} \right|}
\newcommand{\ket}[1]{\left| #1 \right\rangle}

%%%%%%%%%%%%%%%%%%%%%%%%%%%%%%%%%%%%%%%%%%%%%%%%%%%%%%%%%%%%%%%%%%%%%%%%%%%%%%%
%                                  Headings                                   %
%%%%%%%%%%%%%%%%%%%%%%%%%%%%%%%%%%%%%%%%%%%%%%%%%%%%%%%%%%%%%%%%%%%%%%%%%%%%%%%

% This will set fancy headings to the top of the page. The page number will be
% accompanied by the total number of pages. That way, you will know if any page
% is missing.
%
% If you do not want this for your document, you can just use
% ``\pagestyle{plain}``.

\usepackage{scrpage2}

\pagestyle{scrheadings}
\automark{section}
\cfoot{\footnotesize{Seite \thepage\ / \pageref{LastPage}}}
\chead{}
\ihead{}
\ohead{\rightmark}
\setheadsepline{.4pt}

%%%%%%%%%%%%%%%%%%%%%%%%%%%%%%%%%%%%%%%%%%%%%%%%%%%%%%%%%%%%%%%%%%%%%%%%%%%%%%%
%                            Bibliography (BibTeX)                            %
%%%%%%%%%%%%%%%%%%%%%%%%%%%%%%%%%%%%%%%%%%%%%%%%%%%%%%%%%%%%%%%%%%%%%%%%%%%%%%%

\newcommand{\bibliographyfile}{../../zentrale_BibTeX/Central}
\bibliographystyle{apalike2}

%%%%%%%%%%%%%%%%%%%%%%%%%%%%%%%%%%%%%%%%%%%%%%%%%%%%%%%%%%%%%%%%%%%%%%%%%%%%%%%
%                                Abbreviations                                %
%%%%%%%%%%%%%%%%%%%%%%%%%%%%%%%%%%%%%%%%%%%%%%%%%%%%%%%%%%%%%%%%%%%%%%%%%%%%%%%

\newcommand{\dhabk}{\mbox{d.\,h.}}

%%%%%%%%%%%%%%%%%%%%%%%%%%%%%%%%%%%%%%%%%%%%%%%%%%%%%%%%%%%%%%%%%%%%%%%%%%%%%%%
%                                  Licences                                   %
%%%%%%%%%%%%%%%%%%%%%%%%%%%%%%%%%%%%%%%%%%%%%%%%%%%%%%%%%%%%%%%%%%%%%%%%%%%%%%%

\usepackage{ccicons}

\newcommand{\ccbysadetext}{%
	\begin{small}
		Dieses Werk bzw. Inhalt steht unter einer
		\href{http://creativecommons.org/licenses/by-sa/3.0/deed.de}{%
			Creative Commons Namensnennung - Weitergabe unter gleichen
		Bedingungen 3.0 Unported Lizenz}.
	\end{small}
}

\newcommand{\ccbysadetitle}{%
	Lizenz: \href{http://creativecommons.org/licenses/by-sa/3.0/deed.de}
	{CC-BY-SA 3.0 \ccbysa}
}
`` in your main
% document and start with ``\title`` and ``\begin{document}`` afterwards.

% If you need to add additional packages, I recommend not doing this in this
% file, but in your main document. That way, you can just drop in a new
% ``header.tex`` and get all the new commands without having to merge manually.

% Since this file encorporates a CC-BY-SA fragment, this whole files is
% licensed under the CC-BY-SA license.

\documentclass[11pt, ngerman, fleqn, DIV=15, headinclude]{scrartcl}

\usepackage{graphicx}

% Environment to quote the problem. Currently, this is just a new name for the
% quote environment.
\newenvironment{problem}{\begin{quote}}{\end{quote}}

%%%%%%%%%%%%%%%%%%%%%%%%%%%%%%%%%%%%%%%%%%%%%%%%%%%%%%%%%%%%%%%%%%%%%%%%%%%%%%%
%                                Locale, date                                 %
%%%%%%%%%%%%%%%%%%%%%%%%%%%%%%%%%%%%%%%%%%%%%%%%%%%%%%%%%%%%%%%%%%%%%%%%%%%%%%%

\usepackage{babel}
\usepackage[iso]{isodate}

%%%%%%%%%%%%%%%%%%%%%%%%%%%%%%%%%%%%%%%%%%%%%%%%%%%%%%%%%%%%%%%%%%%%%%%%%%%%%%%
%                          Margins and other spacing                          %
%%%%%%%%%%%%%%%%%%%%%%%%%%%%%%%%%%%%%%%%%%%%%%%%%%%%%%%%%%%%%%%%%%%%%%%%%%%%%%%

\usepackage[parfill]{parskip}
\usepackage{setspace}
\usepackage[activate]{microtype}

\setlength{\columnsep}{2cm}

%%%%%%%%%%%%%%%%%%%%%%%%%%%%%%%%%%%%%%%%%%%%%%%%%%%%%%%%%%%%%%%%%%%%%%%%%%%%%%%
%                                    Color                                    %
%%%%%%%%%%%%%%%%%%%%%%%%%%%%%%%%%%%%%%%%%%%%%%%%%%%%%%%%%%%%%%%%%%%%%%%%%%%%%%%

\usepackage[usenames, dvipsnames]{xcolor}

\colorlet{darkred}{red!70!black}
\colorlet{darkblue}{blue!70!black}
\colorlet{darkgreen}{green!40!black}

%%%%%%%%%%%%%%%%%%%%%%%%%%%%%%%%%%%%%%%%%%%%%%%%%%%%%%%%%%%%%%%%%%%%%%%%%%%%%%%
%                         Font and font like settings                         %
%%%%%%%%%%%%%%%%%%%%%%%%%%%%%%%%%%%%%%%%%%%%%%%%%%%%%%%%%%%%%%%%%%%%%%%%%%%%%%%

% This replaces all fonts with Bitstream Charter, Bitstream Vera Sans and
% Bitstream Vera Mono. Math will be rendered in Charter.
\usepackage[charter, greekuppercase=italicized]{mathdesign}
\usepackage{beramono}
\usepackage{berasans}

% Bold, sans-serif tensors. This fragment is taken from “egreg” from
% http://tex.stackexchange.com/a/82747/8945 and licensed under `CC-BY-SA
% <https://creativecommons.org/licenses/by-sa/3.0/>`_.
\usepackage{bm}
\DeclareMathAlphabet{\mathsfit}{\encodingdefault}{\sfdefault}{m}{sl}
\SetMathAlphabet{\mathsfit}{bold}{\encodingdefault}{\sfdefault}{bx}{sl}
\newcommand{\tens}[1]{\bm{\mathsfit{#1}}}

% Bold vectors.
\renewcommand{\vec}[1]{\boldsymbol{#1}}

%%%%%%%%%%%%%%%%%%%%%%%%%%%%%%%%%%%%%%%%%%%%%%%%%%%%%%%%%%%%%%%%%%%%%%%%%%%%%%%
%                               Input encoding                                %
%%%%%%%%%%%%%%%%%%%%%%%%%%%%%%%%%%%%%%%%%%%%%%%%%%%%%%%%%%%%%%%%%%%%%%%%%%%%%%%

\usepackage[T1]{fontenc}
\usepackage[utf8]{inputenc}

%%%%%%%%%%%%%%%%%%%%%%%%%%%%%%%%%%%%%%%%%%%%%%%%%%%%%%%%%%%%%%%%%%%%%%%%%%%%%%%
%                         Hyperrefs and PDF metadata                          %
%%%%%%%%%%%%%%%%%%%%%%%%%%%%%%%%%%%%%%%%%%%%%%%%%%%%%%%%%%%%%%%%%%%%%%%%%%%%%%%

\usepackage{hyperref}
\usepackage{lastpage}

% This sets the author in the properties of the PDF as well. If you want to
% change it, just override it with another ``\hypersetup`` call.
\hypersetup{
	breaklinks=false,
	citecolor=darkgreen,
	colorlinks=true,
	linkcolor=darkblue,
	menucolor=black,
	pdfauthor={Martin Ueding},
	urlcolor=darkblue,
}

%%%%%%%%%%%%%%%%%%%%%%%%%%%%%%%%%%%%%%%%%%%%%%%%%%%%%%%%%%%%%%%%%%%%%%%%%%%%%%%
%                               Math Operators                                %
%%%%%%%%%%%%%%%%%%%%%%%%%%%%%%%%%%%%%%%%%%%%%%%%%%%%%%%%%%%%%%%%%%%%%%%%%%%%%%%

% AMS environments like ``align`` and theorems like ``proof``.
\usepackage{amsmath}
\usepackage{amsthm}

% Common math constructs like partial derivatives.
\usepackage{commath}

% Physical units.
\usepackage[output-decimal-marker={,}]{siunitx}

% Word like operators.
\DeclareMathOperator{\acosh}{arcosh}
\DeclareMathOperator{\arcosh}{arcosh}
\DeclareMathOperator{\arcsinh}{arsinh}
\DeclareMathOperator{\arsinh}{arsinh}
\DeclareMathOperator{\asinh}{arsinh}
\DeclareMathOperator{\card}{card}
\DeclareMathOperator{\csch}{cshs}
\DeclareMathOperator{\diam}{diam}
\DeclareMathOperator{\sech}{sech}
\renewcommand{\Im}{\mathop{{}\mathrm{Im}}\nolimits}
\renewcommand{\Re}{\mathop{{}\mathrm{Re}}\nolimits}

% Fourier transform.
\DeclareMathOperator{\fourier}{\ensuremath{\mathcal{F}}}

% Roman versions of “e” and “i” to serve as Euler's number and the imaginary
% constant.
\newcommand{\ee}{\eup}
\newcommand{\eup}{\mathrm e}
\newcommand{\ii}{\iup}
\newcommand{\iup}{\mathrm i}

% Symbols for the various mathematical fields (natural numbers, integers,
% rational numbers, real numbers, complex numbers).
\newcommand{\C}{\ensuremath{\mathbb C}}
\newcommand{\N}{\ensuremath{\mathbb N}}
\newcommand{\Q}{\ensuremath{\mathbb Q}}
\newcommand{\R}{\ensuremath{\mathbb R}}
\newcommand{\Z}{\ensuremath{\mathbb Z}}

% Shape like operators.
\DeclareMathOperator{\dalambert}{\Box}
\DeclareMathOperator{\laplace}{\bigtriangleup}
\newcommand{\curl}{\vnabla \times}
\newcommand{\divergence}[1]{\inner{\vnabla}{#1}}
\newcommand{\vnabla}{\vec \nabla}

\newcommand{\half}{\frac 12}

% Unit vector (German „Einheitsvektor“).
\newcommand{\ev}{\hat{\vec e}}

% Scientific notation for large numbers.
\newcommand{\e}[1]{\cdot 10^{#1}}

% Mathematician's notation for the inner (scalar, dot) product.
\newcommand{\bracket}[1]{\left\langle #1 \right\rangle}
\newcommand{\inner}[2]{\bracket{#1, #2}}

% Placeholders.
\newcommand{\emesswert}{\del{\messwert \pm \messwert}}
\newcommand{\fehlt}{\textcolor{darkred}{Hier fehlen noch Inhalte.}}
\newcommand{\messwert}{\textcolor{blue}{\square}}
\newcommand{\punkte}{\phantom{xxxxx}}
\newcommand{\punktevon}[1]{\begin{flushright}/ #1\end{flushright}}

% Separator for equations on a single line.
\newcommand{\eqnsep}{,\quad}

% Quantum Mechanics
\newcommand{\braket}[2]{\left\langle #1 \left. \vphantom{#1 #2} \right| #2 \right\rangle}
\newcommand{\braopket}[3]{\left\langle #1 \left. \vphantom{#1 #2 #3} \right| #2 \left. \vphantom{#1 #2 #3} \right| #3 \right\rangle}
\newcommand{\bra}[1]{\left\langle #1 \right|}
\newcommand{\ketbra}[2]{\left| #1 \vphantom{#2} \right\rangle \left\langle #2  \vphantom{#1} \right|}
\newcommand{\ket}[1]{\left| #1 \right\rangle}

%%%%%%%%%%%%%%%%%%%%%%%%%%%%%%%%%%%%%%%%%%%%%%%%%%%%%%%%%%%%%%%%%%%%%%%%%%%%%%%
%                                  Headings                                   %
%%%%%%%%%%%%%%%%%%%%%%%%%%%%%%%%%%%%%%%%%%%%%%%%%%%%%%%%%%%%%%%%%%%%%%%%%%%%%%%

% This will set fancy headings to the top of the page. The page number will be
% accompanied by the total number of pages. That way, you will know if any page
% is missing.
%
% If you do not want this for your document, you can just use
% ``\pagestyle{plain}``.

\usepackage{scrpage2}

\pagestyle{scrheadings}
\automark{section}
\cfoot{\footnotesize{Seite \thepage\ / \pageref{LastPage}}}
\chead{}
\ihead{}
\ohead{\rightmark}
\setheadsepline{.4pt}

%%%%%%%%%%%%%%%%%%%%%%%%%%%%%%%%%%%%%%%%%%%%%%%%%%%%%%%%%%%%%%%%%%%%%%%%%%%%%%%
%                            Bibliography (BibTeX)                            %
%%%%%%%%%%%%%%%%%%%%%%%%%%%%%%%%%%%%%%%%%%%%%%%%%%%%%%%%%%%%%%%%%%%%%%%%%%%%%%%

\newcommand{\bibliographyfile}{../../zentrale_BibTeX/Central}
\bibliographystyle{apalike2}

%%%%%%%%%%%%%%%%%%%%%%%%%%%%%%%%%%%%%%%%%%%%%%%%%%%%%%%%%%%%%%%%%%%%%%%%%%%%%%%
%                                Abbreviations                                %
%%%%%%%%%%%%%%%%%%%%%%%%%%%%%%%%%%%%%%%%%%%%%%%%%%%%%%%%%%%%%%%%%%%%%%%%%%%%%%%

\newcommand{\dhabk}{\mbox{d.\,h.}}

%%%%%%%%%%%%%%%%%%%%%%%%%%%%%%%%%%%%%%%%%%%%%%%%%%%%%%%%%%%%%%%%%%%%%%%%%%%%%%%
%                                  Licences                                   %
%%%%%%%%%%%%%%%%%%%%%%%%%%%%%%%%%%%%%%%%%%%%%%%%%%%%%%%%%%%%%%%%%%%%%%%%%%%%%%%

\usepackage{ccicons}

\newcommand{\ccbysadetext}{%
	\begin{small}
		Dieses Werk bzw. Inhalt steht unter einer
		\href{http://creativecommons.org/licenses/by-sa/3.0/deed.de}{%
			Creative Commons Namensnennung - Weitergabe unter gleichen
		Bedingungen 3.0 Unported Lizenz}.
	\end{small}
}

\newcommand{\ccbysadetitle}{%
	Lizenz: \href{http://creativecommons.org/licenses/by-sa/3.0/deed.de}
	{CC-BY-SA 3.0 \ccbysa}
}


\usepackage{tikz}
\usetikzlibrary{calc}

\newcommand{\themodul}{physik411}
\newcommand{\thegruppe}{Gruppe 2 -- Florian Seidler}
\newcommand{\theuebung}{5}

\ifoot{\footnotesize{Martin Ueding}}
\ihead{\themodul{} -- Übung \theuebung}
\ofoot{\footnotesize{\thegruppe}}

\def\thesubsection{\thesection\alph{subsection}}

\title{\themodul{} -- Übung \theuebung}
\subtitle{\thegruppe}
\author{
	Martin Ueding \footnote{\href{mailto:mu@uni-bonn.de}{mu@uni-bonn.de}}
}

\hypersetup{
	pdftitle={\themodul {} - Übung \theuebung},
}

\begin{document}

\maketitle

\begin{center}
	\ccbysadetitle
\end{center}

\begin{Form}
	\begin{table}[h]
		\centering
		\begin{tabular}{l|c|c|c|c|c}
			Aufgabe
			& \ref 1
			& \ref 2
			& \ref 3
			& \ref 4
			& $\sum$   \\
			\hline
			Punkte
			& \TextField[name=aufgabe1, width=1cm]{} / 7
			& \TextField[name=aufgabe2, width=1cm]{} / 8
			& \TextField[name=aufgabe3, width=1cm]{} / 6
			& \TextField[name=aufgabe3, width=1cm]{} / 12
			& \TextField[name=ergebnis, width=1cm]{} / 33
		\end{tabular}
	\end{table}
\end{Form}

%%%%%%%%%%%%%%%%%%%%%%%%%%%%%%%%%%%%%%%%%%%%%%%%%%%%%%%%%%%%%%%%%%%%%%%%%%%%%%%
%                        Wasserstoffähnliche Systeme                         %
%%%%%%%%%%%%%%%%%%%%%%%%%%%%%%%%%%%%%%%%%%%%%%%%%%%%%%%%%%%%%%%%%%%%%%%%%%%%%%%

\section{Wasserstoffähnliche Systeme}
\label 1

Der Bohr-Radius ist gegeben als:
\[
	a_0 = \frac{4 \piup \varepsilon_0 \hbar^2}{m_\text e e^2}
\]

Daraus folgt, dass:
\[
	a_0 \propto \frac{1}{m_\text{e} Z}
\]

\subsection{Positronium}

Für das Positronium benutze ich die Lösung des Keplerproblems mit
dem Hamilton-Jakobi-Formalismus. Dort wird das Problem im Schwerpunktsystem
betrachtet, wobei eine reduzierte Masse $m'$ um den dann fixen Kern kreist. Die
Zentralkraft zwischen den beiden Massen bleibt erhalten, der Abstand auch.
Diese reduzierte Masse ist, aus den Massen $m$ des Satelliten und $M$ des
Kerns:
\[
	m' = \frac{m M}{m + M}
\]

In diesem Problem ist:
\[
	m' = \half m_\text e
\]

Somit wächst der Radius auf $2a_0$. Die Energie ist gegeben durch:
\[
	E\del{a_0} = -\half \frac{e^2}{4\piup\varepsilon_0 a_0}
\]

Mit dem Radius $2a_0$ erhalte ich die Hälfte der Rydbergenergie,
$\SI{-6.8}\electronvolt$.

\subsection{Myonischer Wasserstoff}

Beim myonischen Wasserstoff benutze ich den gleichen Ansatz, hier ist die reduzierte Masse:
\[
	m' = \frac{m_\text p m_\muup}{m_\text p + m_\muup} = \frac{\num{1836} \cdot 200}{\num{1836} + 200} m_\text e
	= 180 m_\text e
\]

Jetzt ist die Masse also deutlich größer, so dass der Radius kleiner wird:
$a_0 / 180$. Die Energie wächst um den Faktor \num{180} an, also
$\SI{-2448.0}\electronvolt$.

Im 1s-Zustand ist der Zustand:
\[
	\psi(x, t) = R_{1,0}(r) Y_{0, 0}(\theta, \phi)
\]

Dabei ist $Y_{0, 0} = 1$, und $R_{1,0}(r)$ ist gegeben als:
\[
	R_{1,0}(r) = 2 C_0 \exp\del{- \frac{\rho}{2}}
	\eqnsep
	C_0 = \del{\frac Z{a_0}}^{3/2}
	\eqnsep
	\rho = \frac{2Zr}{a_0}
\]

Also:
\[
	\psi(x, t) = 2 \del{\frac Z{a_0}}^{3/2} \exp\del{- \frac{Zr}{a_0}}
\]

Die Wahrscheinlichkeit, dass das Teilchen im Kern ist, ist:
\begin{align*}
	P_\text{im Kern}
	&= 4 \piup \int_0^{r_{\text p}} \dif r \, r^2 \psi^\dagger(x, t) \psi(x, t) \\
	&= 16 \piup \frac{Z^3}{a_0^3} \int_0^{r_{\text p}} \dif r \, r^2 \exp\del{- \frac{2rZ}{a_0}}
\end{align*}

Dieses Integral kann man mit Substitution und dann partieller Integration
lösen. Ich \href{http://www.smbc-comics.com/?id=2861}{mache dies mit
\emph{Mathematica}} und erhalte für das mit $r_\text p =
\SI{0.8}{\femto\meter}$ und $a_0/180$ folgenden Wert für den Innenbereich:
$\SI{2.14e-45}{\meter\cubed}$. Für den gesamten Raum erhalte ich
$\SI{7.98e-38}{\meter\cubed}$. Der Quotient gibt mir die Wahrscheinlichkeit,
dass sich das Myon im Kern aufhält: $\num{2.68e-8}$. Die Einheit für die
Wahrscheinlichkeit erhalte ich, weil ich die Wahrscheinlichkeitsdichte
$\psi^\dagger \psi$ nicht noch durch $\si{\meter\cubed}$ geteilt hatte.

Für den Wasserstoff muss ich in das obige nur $a_0$ anstelle von $a_0/180$
einsetzen und erhalte: $\num{4.55e15}$.

\subsection{Wasserstoffähnliches Uran}

Beim Wasserstoffatom kann der Kern als unendlich schwer angekommen werden, hier
geht das erst recht. Daher brauche ich nur den Effekt zu betrachten, das $Z =
92$ ist. Somit schrumpft der Radius auf $a_0 / 92$, die Energie wächst auf $92
E_\text{Ryd}$.

Mit dem gleichen Ansatz wie oben, nur dass ich noch $Z = 91$ unterbringen,
sowie $r_\text p$ auf $\sqrt[3]{238} \cdot \SI{1.3}{\femto\meter}$ setzen muss,
erhalte ich für die Wahrscheinlichkeit im Kern $\num{0.00231}$.

%%%%%%%%%%%%%%%%%%%%%%%%%%%%%%%%%%%%%%%%%%%%%%%%%%%%%%%%%%%%%%%%%%%%%%%%%%%%%%%
%                        Zeeman-Effekt im Bohr-Modell                         %
%%%%%%%%%%%%%%%%%%%%%%%%%%%%%%%%%%%%%%%%%%%%%%%%%%%%%%%%%%%%%%%%%%%%%%%%%%%%%%%

\section{Zeeman-Effekt im Bohr-Modell}
\label 2

Die Energie eines Dipols $\vec \mu$ in einem magnetischen Feld $\vec B$ ist:
\[
	E = - \inner{\vec \mu}{\vec B}
\]

Mit $\vec \mu = \gamma \vec \ell$ kann ich dies schreiben als:
\[
	E = n B \mu
\]

Die Zentralkraft muss um die Lorenzkraft ergänzt werden:
\[
	m_\text e \frac{v^2}r = \frac{1}{4\piup \varepsilon_0} \frac{e^2}{r^2} + evB
\]

Wenn $\vec L \perp \vec B$ ist, dann wird eine Drehung um $\piup$ keine
Änderung in der Energie haben. Allerdings wirkt dann ein Drehmoment auf das
Atom.

%%%%%%%%%%%%%%%%%%%%%%%%%%%%%%%%%%%%%%%%%%%%%%%%%%%%%%%%%%%%%%%%%%%%%%%%%%%%%%%
%             Spektroskopie der Zeeman- und Isotopieverschiebung              %
%%%%%%%%%%%%%%%%%%%%%%%%%%%%%%%%%%%%%%%%%%%%%%%%%%%%%%%%%%%%%%%%%%%%%%%%%%%%%%%

\section{Spektroskopie der Zeeman- und Isotopieverschiebung}
\label 3

\subsection{Verschiebung des Energieniveaus}

Die Energieverschiebung ist:
\[
	\Deltaup E_\text{Zee}
	= \gamma B m
	= g_\ell B \mu_B m
	= \SI{4.635e-24}\joule
\]

Dies entspricht einem Frequenzunterschied von:
\[
	\Deltaup f = \frac{\Deltaup E}\hbar
	= \SI{6.9951e9}\hertz
\]

Bei einer Frequenz $f = \SI{4.65516e14}\hertz$ entspricht dies einem Wellenlängenunterschied von:
\[
	\Deltaup \lambda = \frac{c}{f+\Deltaup f} - \lambda
	= \SI{0.009678}{\nano\meter}
\]

\subsection{Isotopieverschiebung}

Ähnlich wie in der Aufgabe \ref 1 ändert sich die reduzierte Masse im System. Die Energiedifferenz ist dann:
\[
	\Deltaup E
	= \del{\frac{\num{1836}}{\num{1837}} + \frac{\num{3670}}{\num{3671}}} E_\text{Ryd} \del{\frac{1}{3^2} - \frac{1}{2^2}}
	= \SI{-0.000513703}\electronvolt
\]

Dies konvertiere ich in einen Frequenzunterschied:
\[
	\Deltaup f = \SI{1.24e11}\hertz
\]

Der Übergang entspricht einer Frequenz von:
\[
	f = \SI{4.567e14}\hertz
\]

Und einer Wellenlänge von:
\[
	\lambda = \SI{656}{\nano\meter}
\]

Und einer Wellenlängendifferenz:
\[
	\Deltaup \lambda = \frac{\lambda}f \Deltaup f
	= \SI{1.79e-10}\meter
\]

\subsection{Finesse}

Der freie Spektralbereich sollte gerade die $\SI{6.9951e9}\hertz$ aus dem
ersten Aufgabenteil sein. Die Finesse sagt eigentlich nur etwas über die
Schärfe der Peaks aus, also $\Deltaup \lambda / \deltaup \lambda$. Dabei habe
ich nur $\Deltaup \lambda / \lambda$.

%%%%%%%%%%%%%%%%%%%%%%%%%%%%%%%%%%%%%%%%%%%%%%%%%%%%%%%%%%%%%%%%%%%%%%%%%%%%%%%
%                     Spin-Bahn-Kopplung und $g_j$-Faktor                     %
%%%%%%%%%%%%%%%%%%%%%%%%%%%%%%%%%%%%%%%%%%%%%%%%%%%%%%%%%%%%%%%%%%%%%%%%%%%%%%%

\section{Spin-Bahn-Kopplung und $g_j$-Faktor}
\label 4

\subsection{Komponenten und Digramme}

Bei $\ell = 1$ ist die Länge des Drehimpulses $\sqrt{1(1+1)} = \sqrt 2$. Mit $s
= 1/2$ ist die Länge des Spins $\sqrt 3/2$. Die $z$-Komponenten $m_l$ und $m_s$
sind $-1, 0, 1$ bzw. $-\half, \half$. Der Polarwinkel $\theta$ mit der
$z$-Achse ist dann gegeben als:
\[
	\cos\del{\theta_\ell} = \frac{m_l}{\sqrt 2} = \pm 1, 0
	\eqnsep
	\cos\del{\theta_s} = \frac{m_s}{\sqrt{3}/2} = \pm \frac{1}{\sqrt 3}
\]

Die Winkel $\SI{90}\degree - \theta$ sind dann:
\[
	\theta_\ell = \pm \SI{45}\degree
	\eqnsep
	\theta_s = \pm \SI{35.2644}\degree
\]

Alle signifikant verschiedenen Vektordiagramme, also solche, die nicht durch
Spiegellung oder Drehung aus den anderen hervorgehen können, sind in den
Abbildungen \ref{fig:4/a/1}, \ref{fig:4/a/2}, \ref{fig:4/a/3}, \ref{fig:4/a/4},
\ref{fig:4/a/5} und \ref{fig:4/a/6} gezeigt. In Abbildung \ref{fig:4/a/alle}
sind alle Vektoren $\vec \ell$ und $\vec \sigma$ gezeigt.

\begin{figure}
	\centering
	\begin{tikzpicture}[scale=4]
		\coordinate (l1) at (45:1.41421);
		\coordinate (l0) at (0:1.41421);
		\coordinate (l-1) at (-45:1.41421);

		\coordinate (s+l) at (144.736:0.866025);
		\coordinate (s+r) at (35.2644:0.866025);
		\coordinate (s-l) at (-144.736:0.866025);
		\coordinate (s-r) at (-35.2644:0.866025);

		\draw[->] (0, 0) -- (l1) node[midway, sloped, below] {$\vec \ell$, $m_\ell=1$};
		\draw[->] (0, 0) -- (l0) node[midway, sloped, below] {$\vec \ell$, $m_\ell=0$};

		\draw[->] (l0) -- ($(l0)+(s+r)$) node[midway, sloped, above] {$\vec \sigma$, $m_s=\half$};
		\draw[->] (l0) -- ($(l0)+(s+l)$) node[midway, sloped, above] {$\vec \sigma$, $m_s=\half$};
		\draw[->] (l1) -- ($(l1)+(s+r)$) node[midway, sloped, above] {$\vec \sigma$, $m_s=\half$};
		\draw[->] (l1) -- ($(l1)+(s+l)$) node[midway, sloped, above] {$\vec \sigma$, $m_s=\half$};

		\draw[->] (l1) -- ($(l1)+(s-r)$) node[midway, sloped, above] {$\vec \sigma$, $m_s=-\half$};
		\draw[->] (l1) -- ($(l1)+(s-l)$) node[midway, sloped, above] {$\vec \sigma$, $m_s=-\half$};
	\end{tikzpicture}
	\caption{%
		Alle Vektoren $\vec l$ und $\vec \sigma$ in einem Diagramm. Dabei habe
		ich die Fälle, in denen alle $z$-Komponenten umgekehrt sind,
		weggelassen, da sie letztlich das gleiche darstellen.
	}
	\label{fig:4/a/alle}
\end{figure}

\begin{figure}
	\centering
	\begin{tikzpicture}[scale=4]
		\coordinate (l1) at (45:1.41421);
		\coordinate (l0) at (0:1.41421);
		\coordinate (l-1) at (-45:1.41421);

		\coordinate (s+l) at (144.736:0.866025);
		\coordinate (s+r) at (35.2644:0.866025);
		\coordinate (s-l) at (-144.736:0.866025);
		\coordinate (s-r) at (-35.2644:0.866025);

		\draw[->] (0, 0) -- (l1) node[midway, sloped, below] {$\vec \ell$, $m_\ell=1$};
		\draw[->] (l1) -- ($(l1)+(s+l)$) node[midway, sloped, above] {$\vec \sigma$, $m_s=\half$};
		\draw[->] (0, 0) -- ($(l1)+(s+l)$) node[midway, sloped, above] {$\vec j$};
	\end{tikzpicture}
	\caption{%
		$m_\ell = 1$ und $m_s = \half$
	}
	\label{fig:4/a/1}
\end{figure}

\begin{figure}
	\centering
	\begin{tikzpicture}[scale=4]
		\coordinate (l1) at (45:1.41421);
		\coordinate (l0) at (0:1.41421);
		\coordinate (l-1) at (-45:1.41421);

		\coordinate (s+l) at (144.736:0.866025);
		\coordinate (s+r) at (35.2644:0.866025);
		\coordinate (s-l) at (-144.736:0.866025);
		\coordinate (s-r) at (-35.2644:0.866025);

		\draw[->] (0, 0) -- (l1) node[midway, sloped, above] {$\vec \ell$, $m_\ell=1$};
		\draw[->] (l1) -- ($(l1)+(s+r)$) node[midway, sloped, above] {$\vec \sigma$, $m_s=\half$};
		\draw[->] (0, 0) -- ($(l1)+(s+r)$) node[midway, sloped, below] {$\vec j$};
	\end{tikzpicture}
	\caption{%
		$m_\ell = 1$ und $m_s = \half$
	}
	\label{fig:4/a/2}
\end{figure}

\begin{figure}
	\centering
	\begin{tikzpicture}[scale=4]
		\coordinate (l1) at (45:1.41421);
		\coordinate (l0) at (0:1.41421);
		\coordinate (l-1) at (-45:1.41421);

		\coordinate (s+l) at (144.736:0.866025);
		\coordinate (s+r) at (35.2644:0.866025);
		\coordinate (s-l) at (-144.736:0.866025);
		\coordinate (s-r) at (-35.2644:0.866025);

		\draw[->] (0, 0) -- (l1) node[midway, sloped, below] {$\vec \ell$, $m_\ell=1$};
		\draw[->] (l1) -- ($(l1)+(s-l)$) node[midway, sloped, above] {$\vec \sigma$, $m_s=-\half$};
		\draw[->] (0, 0) -- ($(l1)+(s-l)$) node[midway, sloped, above] {$\vec j$};
	\end{tikzpicture}
	\caption{%
		$m_\ell = 1$ und $m_s = -\half$
	}
	\label{fig:4/a/3}
\end{figure}

\begin{figure}
	\centering
	\begin{tikzpicture}[scale=4]
		\coordinate (l1) at (45:1.41421);
		\coordinate (l0) at (0:1.41421);
		\coordinate (l-1) at (-45:1.41421);

		\coordinate (s+l) at (144.736:0.866025);
		\coordinate (s+r) at (35.2644:0.866025);
		\coordinate (s-l) at (-144.736:0.866025);
		\coordinate (s-r) at (-35.2644:0.866025);

		\draw[->] (0, 0) -- (l1) node[midway, sloped, above] {$\vec \ell$, $m_\ell=1$};
		\draw[->] (l1) -- ($(l1)+(s-r)$) node[midway, sloped, above] {$\vec \sigma$, $m_s=-\half$};
		\draw[->] (0, 0) -- ($(l1)+(s-r)$) node[midway, sloped, below] {$\vec j$};
	\end{tikzpicture}
	\caption{%
		$m_\ell = 1$ und $m_s = -\half$
	}
	\label{fig:4/a/4}
\end{figure}

\begin{figure}
	\centering
	\begin{tikzpicture}[scale=4]
		\coordinate (l1) at (45:1.41421);
		\coordinate (l0) at (0:1.41421);
		\coordinate (l-1) at (-45:1.41421);

		\coordinate (s+l) at (144.736:0.866025);
		\coordinate (s+r) at (35.2644:0.866025);
		\coordinate (s-l) at (-144.736:0.866025);
		\coordinate (s-r) at (-35.2644:0.866025);

		\draw[->] (0, 0) -- (l0) node[midway, sloped, below] {$\vec \ell$, $m_\ell=0$};
		\draw[->] (l0) -- ($(l0)+(s+l)$) node[midway, sloped, above] {$\vec \sigma$, $m_s=\half$};
		\draw[->] (0, 0) -- ($(l0)+(s+l)$) node[midway, sloped, above] {$\vec j$};
	\end{tikzpicture}
	\caption{%
		$m_\ell = 0$ und $m_s = \half$
	}
	\label{fig:4/a/5}
\end{figure}

\begin{figure}
	\centering
	\begin{tikzpicture}[scale=4]
		\coordinate (l1) at (45:1.41421);
		\coordinate (l0) at (0:1.41421);
		\coordinate (l-1) at (-45:1.41421);

		\coordinate (s+l) at (144.736:0.866025);
		\coordinate (s+r) at (35.2644:0.866025);
		\coordinate (s-l) at (-144.736:0.866025);
		\coordinate (s-r) at (-35.2644:0.866025);

		\draw[->] (0, 0) -- (l0) node[midway, sloped, below] {$\vec \ell$, $m_\ell=0$};
		\draw[->] (l0) -- ($(l0)+(s+r)$) node[midway, sloped, below] {$\vec \sigma$, $m_s=\half$};
		\draw[->] (0, 0) -- ($(l0)+(s+r)$) node[midway, sloped, above] {$\vec j$};
	\end{tikzpicture}
	\caption{%
		$m_\ell = 0$ und $m_s = \half$
	}
	\label{fig:4/a/6}
\end{figure}

\subsection{Magnetische Momente}

Für die magnetischen Momente gilt:
\[
	\vec \mu_\ell = - \mu_B \vec \ell
	\eqnsep
	\vec \mu_s = - 2 \mu_B \vec \sigma
\]

Das magnetische Moment des Spins ist also im Vergleich zu den Drehmomenten
doppelt so groß wie das magnetische Moment des Spins. Daher ist die vektorielle
Summe $\vec \mu_j$ auch anders als $\vec j$.

In Abbildung \ref{fig:4/a/1} habe ich die magnetischen Momente eingetragen, das
Resultat ist in Abbildung \ref{fig:4/b/1}.

\begin{figure}
	\centering
	\begin{tikzpicture}[scale=4]
		\coordinate (l1) at (45:1.41421);
		\coordinate (l0) at (0:1.41421);
		\coordinate (l-1) at (-45:1.41421);

		\coordinate (mul1) at (l1);

		\coordinate (s+l) at (144.736:0.866025);
		\coordinate (s+r) at (35.2644:0.866025);
		\coordinate (s-l) at (-144.736:0.866025);
		\coordinate (s-r) at (-35.2644:0.866025);

		\coordinate (mus+l) at (144.736:1.73205);

		\draw[->, dotted] (0, 0) -- (l1) node[midway, sloped, below] {};
		\draw[->, dotted] (l1) -- ($(l1)+(s+l)$) node[midway, sloped, above] {};
		\draw[->] (0, 0) -- ($(l1)+(s+l)$) node[midway, sloped, above] {$\vec j$};

		\draw[->] (0, 0) -- (mul1) node[midway, sloped, below] {$\vec \mu_\ell$};
		\draw[->] (mul1) -- ($(mul1)+(mus+l)$) node[midway, sloped, above] {$\vec \mu_s$};
		\draw[->] (0, 0) -- ($(mul1)+(mus+l)$) node[midway, sloped, above] {$\vec \mu_j$};
	\end{tikzpicture}
	\caption{%
		$m_\ell = 1$ und $m_s = \half$
	}
	\label{fig:4/b/1}
\end{figure}

In dieser Abbildung ist dann auch zu sehen, dass $\vec j \nparallel \vec\mu_j$
gilt.

\subsection{Präzession von $\vec \mu_j$ um $\vec j$}

\fehlt

\subsection{$g$-Faktoren}

\fehlt

\subsection{externes Magnetfeld}

\fehlt

%%%%%%%%%%%%%%%%%%%%%%%%%%%%%%%%%%%%%%%%%%%%%%%%%%%%%%%%%%%%%%%%%%%%%%%%%%%%%%%
%                                    Ende                                     %
%%%%%%%%%%%%%%%%%%%%%%%%%%%%%%%%%%%%%%%%%%%%%%%%%%%%%%%%%%%%%%%%%%%%%%%%%%%%%%%

\IfFileExists{\bibliographyfile}{
	%\bibliography{\bibliographyfile}
}{}

\end{document}

% vim: spell spelllang=de
