% Copyright © 2013 Martin Ueding <dev@martin-ueding.de>
%
\input{header.tex}

\usepackage{tikz}

\newcommand{\themodul}{physik411}
\newcommand{\thegruppe}{Gruppe 2 -- Steffen Urban}
\newcommand{\theuebung}{2}

\ifoot{\footnotesize{Martin Ueding}}
\ihead{\themodul{} -- Übung \theuebung}
\ofoot{\footnotesize{\thegruppe}}

\def\thesubsection{\thesection\alph{subsection}}

\title{\themodul{} -- Übung \theuebung \\ \vspace{0.5cm} \large{\thegruppe}}

\author{
	Martin Ueding \footnote{\href{mailto:mu@uni-bonn.de}{mu@uni-bonn.de}}
}

\hypersetup{
	pdftitle={\themodul {} - Übung \theuebung},
}

\begin{document}

\maketitle

\begin{table}[h]
	\centering
	\begin{tabular}{l|c|c|c|c|c}
		Aufgabe
		& \ref 1
		& \ref 2
		& \ref 3
		& \ref 4
		& $\sum$   \\
		\hline
		Punkte
		& \punkte / 4
		& \punkte / 4
		& \punkte / 11
		& \punkte / 10
		& \punkte / 29
	\end{tabular}
\end{table}

%%%%%%%%%%%%%%%%%%%%%%%%%%%%%%%%%%%%%%%%%%%%%%%%%%%%%%%%%%%%%%%%%%%%%%%%%%%%%%%
%                      Quanteneffizienz einer Photodiode                      %
%%%%%%%%%%%%%%%%%%%%%%%%%%%%%%%%%%%%%%%%%%%%%%%%%%%%%%%%%%%%%%%%%%%%%%%%%%%%%%%

\section{Quanteneffizienz einer Photodiode}
\label 1

Mit der Photonenzahl $m$ und der Elektronenzahl $n$:
\[
	S_\lambda
	= \frac{I}{P}
	= \frac{I \lambda t}{m h c}
	= \frac{Q \lambda}{m h c}
	= \frac{n e \lambda}{m h c}
	\iff
	\frac nm = \frac{S_\lambda h c}{e \lambda}
	\implies
	\frac nm = 0.904
\]

Also \SI{90.4}{\percent} der eintreffenden Photonen lösen ein Elektron aus.

%%%%%%%%%%%%%%%%%%%%%%%%%%%%%%%%%%%%%%%%%%%%%%%%%%%%%%%%%%%%%%%%%%%%%%%%%%%%%%%
%                    Unschärferelation im Kastenpotential                    %
%%%%%%%%%%%%%%%%%%%%%%%%%%%%%%%%%%%%%%%%%%%%%%%%%%%%%%%%%%%%%%%%%%%%%%%%%%%%%%%

\section{Unschärferelation im Kastenpotential}
\label 2

\[
	\Deltaup x \Deltaup p \geq \frac{\hbar}{2}
	\iff
	\Deltaup p \geq \frac{\hbar}{2a}
	\iff
	\Deltaup v \geq \frac{\hbar}{2am}
	\iff
	\del{\Deltaup v}^2 \geq \frac{\hbar^2}{4a^2m^2}
	\iff
	\Deltaup E = \half m \del{\Deltaup v}^2 \geq \frac{\hbar^2}{8a^2m}
\]

Auf dem vorherigen Aufgabenzettel habe ich die Energie nicht richtig bestimmen
können, daher kann ich sie jetzt nicht vergleichen.

%%%%%%%%%%%%%%%%%%%%%%%%%%%%%%%%%%%%%%%%%%%%%%%%%%%%%%%%%%%%%%%%%%%%%%%%%%%%%%%
%            Quasiklassische Zustände im harmonischen Oszillator             %
%%%%%%%%%%%%%%%%%%%%%%%%%%%%%%%%%%%%%%%%%%%%%%%%%%%%%%%%%%%%%%%%%%%%%%%%%%%%%%%

\section{Quasiklassische Zustände im harmonischen Oszillator}
\label 3

\subsection{Aufenthaltswahrscheinlichkeit}

Die Aufenthaltswahrscheinlichkeit ist:
\begin{align*}
	\abs{\Psi(x, t)}^2 &= \frac{1}{x_0 2^n n!} \frac{1}{\sqrt \piup} H_n\del{\frac x{x_0}} \exp\del{-\frac{x^2}{x_0^2}}
\end{align*}

Allerdings hängt dies nicht mehr von der Zeit $t$ sondern nur noch vom Ort $x$
ab. Somit ist die Aufenthaltswahrscheinlichkeit als Funktion der Zeit $t$ nur
eine konstante Funktion. Und dies kann es wohl nicht sein, oder?

Für verschiedene $n$ ist $\abs{\Psi_n(x,t)}^2$ in Abbildung \ref{fig:n}
geplottet.

\begin{figure}
	\centering
	\includegraphics[width=0.5\textwidth]{H3-n.pdf}
	\caption{Plot von $\abs{\Psi_n(x,t)}^2$ für $n = 0, \ldots, 3$. Horizontale Achse ist $x$, vertikale Achse ist die Wahrscheinlichkeitsdichte.}
	\label{fig:n}
\end{figure}

\subsection{Überlagerungszustand}

Zuerst schaue ich, ob die Wellenfunktion $\Psi_0$ schon normiert ist:
\begin{align*}
	\braket{\Psi_0}{\Psi_0}
	&= \int_{-\infty}^\infty \dif x \, \frac{1}{x_0} \frac{1}{\sqrt\piup} \exp\del{-\frac{x^2}{x_0^2}} \\
	&= \frac{1}{x_0} \frac{1}{\sqrt\piup} \int_{-\infty}^\infty \dif x \, \exp\del{-\frac{x^2}{x_0^2}} \\
	&= \frac{1}{\sqrt\piup} \int_{-\infty}^\infty \dif x' \, \exp\del{-x'^2} \\
	&= 1
\end{align*}

$\Psi_0$ ist also schon normiert. Ich gehe davon aus, das die restlichen
$\Psi_n$ auch schon normiert sind.

Nun betrachte ich die Überlagerung:
\[
	\ket{\Phi} = \ket{\Psi_0} + \epsilon \ket{\Psi_1}
\]

Für den Zustand $\ket\Phi$ bestimme ich nun die Wahrscheinlichkeitsdichte:
\begin{align*}
	\abs{\Phi(x, t)}^2
	&= \Phi^*(x, t) \Phi(x, t) \\
	&= \del{\Psi_0^*(x, t) + \epsilon \Psi_1^*(x, t)} \del{\Psi_0(x, t) + \epsilon \Psi_1(x, t)} \\
	&= \Psi_0^*(x, t) \Psi_0(x, t) + \epsilon \Psi_0^*(x, t) \Psi_0(x, t) + \epsilon \Psi_1^*(x, t) \Psi_1(x, t) + \mathcal O\del{\epsilon^2} \\
	&= \Psi_0^*(x, t) \Psi_0(x, t) + \epsilon \Psi_1^*(x, t) \Psi_0(x, t) + \epsilon \Psi_0^*(x, t) \Psi_1(x, t) + \mathcal O\del{\epsilon^2} \\
	&= \frac{1}{x_0 \sqrt\piup} \exp\del{-\frac{x^2}{x_0^2}} + \frac{\epsilon}{x_0\sqrt\piup} \frac{2x}{x_0} \exp\del{-\frac{x^2}{x_0^2}} \del{\exp\del{-\ii (0-1)\omega t} + \exp\del{-\ii (1-0)\omega t}} \\
	&= \frac{1}{x_0 \sqrt\piup} \exp\del{-\frac{x^2}{x_0^2}} + \frac{\epsilon}{x_0\sqrt\piup} \frac{4x}{x_0} \exp\del{-\frac{x^2}{x_0^2}} \cos\del{\omega t} \\
	\bracket x
	&= \int_{-\infty}^\infty \dif x \, x \del{\frac{1}{x_0 \sqrt\piup} \exp\del{-\frac{x^2}{x_0^2}} + \frac{\epsilon}{x_0\sqrt\piup} \frac{4x}{x_0} \exp\del{-\frac{x^2}{x_0^2}}} \cos\del{\omega t} \\
	\intertext{%
		Der erste Summand ist einer ungerade Funktion, somit verschwindet das
		Integral. Es bleibt der zweite Summand.
	}
	&= \int_{-\infty}^\infty \dif x \, x \frac{\epsilon}{x_0\sqrt\piup} \frac{4x}{x_0} \exp\del{-\frac{x^2}{x_0^2}} \cos\del{\omega t} \\
	&= \frac{4\epsilon}{\sqrt\piup} \int_{-\infty}^\infty \dif x \, \frac{x^2}{x_0^2} \exp\del{-\frac{x^2}{x_0^2}} \cos\del{\omega t} \\
	\intertext{%
		Substituiere $x' := x/x_0$.
	}
	&= \frac{4\epsilon x_0}{\sqrt\piup} \int_{-\infty}^\infty \dif x' \, x'^2 \exp\del{-x'^2} \cos\del{\omega t} \\
	\intertext{%
		Wende das gegebene Integral an.
	}
	&= 2 \epsilon x_0 \cos\del{\omega t}
\end{align*}

Die Teilchenposition verschiebt sich nun periodisch etwas zur Seite, wie auch
in Abbildung \ref{fig:1} zu sehen ist.

\subsection{Plot}

In den Abbildungen \ref{fig:0}, \ref{fig:1} und \ref{fig:3} ist die
Zeitentwicklung von $\rho_i(x, t) := \abs{\Phi_i(x, t)}^2$ geplottet, auf der
horizontalen Achse $x$ und auf der vertikalen Achse $t$. Die Farbe gibt $\rho$
an.

In Abbildung \ref{fig:0} ist $\ket{\Phi_0} = \ket{\Psi_0}$, in Abbildung
\ref{fig:1} ist $\ket{\Phi_1} = \ket{\Psi_0} + \num{0.1} \ket{\Psi_1}$
geplottet. Und in Abbildung \ref{fig:3} $\ket{\Psi_3} := \ket{\Psi_0} +
\num{0.1} \ket{\Psi_1} + \num{0.1} \ket{\Psi_2} + \num{0.1} \ket{\Psi_3}$.

\begin{figure}
	\begin{minipage}[t]{0.3\linewidth}
		\includegraphics[width=\textwidth]{H3-0.pdf}
		\caption{Dichte $\rho_0$}
		\label{fig:0}
	\end{minipage}
	\hfill
	\begin{minipage}[t]{0.3\linewidth}
		\includegraphics[width=\textwidth]{H3-1.pdf}
		\caption{Dichte $\rho_1$}
		\label{fig:1}
	\end{minipage}
	\hfill
	\begin{minipage}[t]{0.3\linewidth}
		\includegraphics[width=\textwidth]{H3-3.pdf}
		\caption{Dichte $\rho_3$}
		\label{fig:3}
	\end{minipage}
\end{figure}

Dieses Teilchen verhält sich wie mehrere Teilchen, die zusammen eine Schwebung
erzeugen. Das Maximum wandert schon fast chaotisch in Abbildung \ref{fig:3}.
Ein klassischer harmonischer Oszillator hat nur eine Eigenfrequenz und kann
sich nicht derart überlagern. Erst, wenn mehrere Massen eingebracht werden,
kann es zu Schwebungen kommen.

%%%%%%%%%%%%%%%%%%%%%%%%%%%%%%%%%%%%%%%%%%%%%%%%%%%%%%%%%%%%%%%%%%%%%%%%%%%%%%%
%                  Das Wasserstoffatom, klassisch betrachtet                  %
%%%%%%%%%%%%%%%%%%%%%%%%%%%%%%%%%%%%%%%%%%%%%%%%%%%%%%%%%%%%%%%%%%%%%%%%%%%%%%%

\section{Das Wasserstoffatom, klassisch betrachtet}
\label 4

\subsection{kinetische, potentielle und gesamte Energie}

Vom statischen Bezugssystem aus gesehen muss für eine Kreisbahn die benötigte
Zentripetalkraft durch die elektrische Anziehung geliefert werden:
\begin{align*}
	\frac{1}{4 \piup \varepsilon_0} \frac{e^2}{r^2} &= m_e \frac{v^2}{r} \\
	\half m_e v^2
	&= \frac{1}{8 \piup \varepsilon_0} \frac{e^2}{r} \\
	E_\text{kin}(r)
	&= \frac{1}{8 \piup \varepsilon_0} \frac{e^2}{r} \\
	E_\text{kin}(r)
	&= \SI{1.154e-28}{\joule\meter} \cdot \frac 1r
\end{align*}

Die potentielle Energie ist für $\lim_{r \to \infty} E_\text{pot}(r) = 0$:
\begin{align*}
	E_\text{pot}(r) &= - \frac{1}{4 \piup \varepsilon_0} \frac{e^2}{r} \\
	E_\text{pot}(r) &= \SI{-2.307e-28}{\joule\meter} \cdot \frac 1r
\end{align*}

Zusammen ist die Energie:
\begin{align*}
	E(r) &= - \frac{1}{8 \piup \varepsilon_0} \frac{e^2}{r} \\
	E(r) &= \SI{-1.154e-28}{\joule\meter} \cdot \frac 1r
\end{align*}

\subsection{Umlaufradius und Bahngeschwindigkeit}

\[
	E(r) = \SI{-13.6}{\electronvolt}
	\implies
	r = \SI{5.29e-11}{\meter}
\]

Dies deckt sich gut mit dem Atomdurchmesser von $\SI{e-10}{\meter}$.

Die Bahngeschwindigkeit erhalte ich mit:
\[
	E_\text{kin}(r) = 1/2 m_e v^2
	\implies
	v = \SI{2.19e6}{\meter\per\second}
\]

\subsection{Abgestrahlte Energie}

\[
	\beta = \num{0.00730}
	\eqnsep
	\gamma = \num{1.00003}
	\eqnsep
	\Deltaup E = \num{2446} \frac{e^2}{\varepsilon_0 m_e} = \SI{7.090e-24}\joule
\]

Mit einer Energie von $\abs E = \SI{13.6}\electronvolt$ gibt es $n =
\num{307000}$ Umläufe, bevor die komplette Energie verbraucht ist. Die Zeit
$t$, die für diese $n$ Umläufe gebraucht wird, ist:
\[
	t = \frac{2 n \piup r}v = \SI{4.67e-11}\second
\]

Also auf einer sehr kleinen Zeitskala, im Vergleich zum Alter der Erde, würden
die Elektronen ihre komplette Energie abstrahlen und auf den Kern fallen. Die
Atome würden dann eventuell zu puren Neutronen fusionieren und die Erde würde
erstmal zu einem Neutronenstern. Jedenfalls stimmt dies absolut nicht mit der
Beobachtung überein.

\subsection{Dilemma}

Die Lösung, die zur Quantenmechanik geführt hat, ist der Vorschlag, dass
Elektronen auf ihren Bahnen um das Atom einfach keine Bremsstrahlung abgeben.

Es könnte auch sein, dass die Elektronen im thermischen Gleichgewicht mit allen
anderen Atomen stehen und sich so der Strahlungsverlust ausgleicht. Allerdings
sind wohl im Universum auch einzelne Atome, so dass diese vielleicht
kollabieren. Keine wirkliche Lösung.

%%%%%%%%%%%%%%%%%%%%%%%%%%%%%%%%%%%%%%%%%%%%%%%%%%%%%%%%%%%%%%%%%%%%%%%%%%%%%%%
%                                    Ende                                     %
%%%%%%%%%%%%%%%%%%%%%%%%%%%%%%%%%%%%%%%%%%%%%%%%%%%%%%%%%%%%%%%%%%%%%%%%%%%%%%%

\IfFileExists{\bibliographyfile}{
	%\bibliography{\bibliographyfile}
}{}

\end{document}

% vim: spell spelllang=de
