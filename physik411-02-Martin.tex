% Copyright © 2013 Martin Ueding <dev@martin-ueding.de>
%
\input{header.tex}

\usepackage{tikz}

\newcommand{\themodul}{physik411}
\newcommand{\thegruppe}{Gruppe 2 -- Steffen Urban}
\newcommand{\theuebung}{2}

\ifoot{\footnotesize{Martin Ueding}}
\ihead{\themodul{} -- Übung \theuebung}
\ofoot{\footnotesize{\thegruppe}}

\def\thesubsection{\thesection\alph{subsection}}

\title{\themodul{} -- Übung \theuebung \\ \vspace{0.5cm} \large{\thegruppe}}

\author{
	Martin Ueding \\ \small{\href{mailto:mu@uni-bonn.de}{mu@uni-bonn.de}}
}

\hypersetup{
	pdftitle={\themodul {} - Übung \theuebung},
}

\begin{document}

\maketitle

\begin{table}[h]
	\centering
	\begin{tabular}{l|c|c|c|c|c}
		Aufgabe
		& \ref 1
		& \ref 2
		& \ref 3
		& \ref 4
		& $\sum$   \\
		\hline
		Punkte
		& \punkte / 4
		& \punkte / 4
		& \punkte / 11
		& \punkte / 10
		& \punkte / 29
	\end{tabular}
\end{table}

%%%%%%%%%%%%%%%%%%%%%%%%%%%%%%%%%%%%%%%%%%%%%%%%%%%%%%%%%%%%%%%%%%%%%%%%%%%%%%%
%                      Quanteneffizienz einer Photodiode                      %
%%%%%%%%%%%%%%%%%%%%%%%%%%%%%%%%%%%%%%%%%%%%%%%%%%%%%%%%%%%%%%%%%%%%%%%%%%%%%%%

\section{Quanteneffizienz einer Photodiode}
\label 1

Mit der Photonenzahl $m$ und der Elektronenzahl $n$:
\[
	S_\lambda
	= \frac{I}{P}
	= \frac{I \lambda t}{m h c}
	= \frac{Q \lambda}{m h c}
	= \frac{n e \lambda}{m h c}
	\iff
	\frac nm = \frac{S_\lambda h c}{e \lambda}
	\implies
	\frac nm = 0.904
\]

Also \SI{90.4}{\percent} der eintreffenden Photonen lösen ein Elektron aus.

%%%%%%%%%%%%%%%%%%%%%%%%%%%%%%%%%%%%%%%%%%%%%%%%%%%%%%%%%%%%%%%%%%%%%%%%%%%%%%%
%                    Unschärferelation im Kastenpotential                    %
%%%%%%%%%%%%%%%%%%%%%%%%%%%%%%%%%%%%%%%%%%%%%%%%%%%%%%%%%%%%%%%%%%%%%%%%%%%%%%%

\section{Unschärferelation im Kastenpotential}
\label 2

\[
	\Deltaup x \Deltaup p \geq \frac{\hbar}{2}
	\iff
	\Deltaup p \geq \frac{\hbar}{2a}
	\iff
	\Deltaup v \geq \frac{\hbar}{2am}
	\iff
	\del{\Deltaup v}^2 \geq \frac{\hbar^2}{4a^2m^2}
	\iff
	\Deltaup E = \half m \del{\Deltaup v}^2 \geq \frac{\hbar^2}{8a^2m}
\]

\fehlt

%%%%%%%%%%%%%%%%%%%%%%%%%%%%%%%%%%%%%%%%%%%%%%%%%%%%%%%%%%%%%%%%%%%%%%%%%%%%%%%
%            Quasiklassische Zustände im harmonischen Oszillator             %
%%%%%%%%%%%%%%%%%%%%%%%%%%%%%%%%%%%%%%%%%%%%%%%%%%%%%%%%%%%%%%%%%%%%%%%%%%%%%%%

\section{Quasiklassische Zustände im harmonischen Oszillator}
\label 3

\fehlt

%%%%%%%%%%%%%%%%%%%%%%%%%%%%%%%%%%%%%%%%%%%%%%%%%%%%%%%%%%%%%%%%%%%%%%%%%%%%%%%
%                  Das Wasserstoffatom, klassisch betrachtet                  %
%%%%%%%%%%%%%%%%%%%%%%%%%%%%%%%%%%%%%%%%%%%%%%%%%%%%%%%%%%%%%%%%%%%%%%%%%%%%%%%

\section{Das Wasserstoffatom, klassisch betrachtet}
\label 4

\subsection{kinetische, potentielle und gesamte Energie}

Vom statischen Bezugssystem aus gesehen muss für eine Kreisbahn die benötigte Zentripetalkraft durch die elektrische Anziehung geliefert werden:
\begin{align*}
	\frac{1}{4 \piup \varepsilon_0} \frac{e^2}{r^2} &= m_e \frac{v^2}{r} \\
	\half m_e v^2
	&= \frac{1}{8 \piup \varepsilon_0} \frac{e^2}{r} \\
	E_\text{kin}(r)
	&= \frac{1}{8 \piup \varepsilon_0} \frac{e^2}{r} \\
	E_\text{kin}(r)
	&= \SI{1.154e-28}{\joule\meter} \cdot \frac 1r
\end{align*}

Die potentielle Energie ist für $\lim_{r \to \infty} E_\text{pot}(r) = 0$:
\begin{align*}
	E_\text{pot}(r) &= - \frac{1}{4 \piup \varepsilon_0} \frac{e^2}{r} \\
	E_\text{pot}(r) &= \SI{-2.307e-28}{\joule\meter} \cdot \frac 1r
\end{align*}

Zusammen ist die Energie:
\begin{align*}
	E(r) &= - \frac{1}{8 \piup \varepsilon_0} \frac{e^2}{r} \\
	E(r) &= \SI{-1.154e-28}{\joule\meter} \cdot \frac 1r
\end{align*}

\punktevon{3}

\subsection{Umlaufradius und Bahngeschwindigkeit}

\[
	E(r) = \SI{-13.6}{\electronvolt}
	\implies
	r = \SI{5.29e-11}{\meter}
\]

\punktevon{2}

%%%%%%%%%%%%%%%%%%%%%%%%%%%%%%%%%%%%%%%%%%%%%%%%%%%%%%%%%%%%%%%%%%%%%%%%%%%%%%%
%                                    Ende                                     %
%%%%%%%%%%%%%%%%%%%%%%%%%%%%%%%%%%%%%%%%%%%%%%%%%%%%%%%%%%%%%%%%%%%%%%%%%%%%%%%

\IfFileExists{\bibliographyfile}{
	%\bibliography{\bibliographyfile}
}{}

\end{document}

% vim: spell spelllang=de
