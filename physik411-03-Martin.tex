% Copyright © 2013 Martin Ueding <dev@martin-ueding.de>
%
\input{header.tex}

\newcommand{\themodul}{physik411}
\newcommand{\thegruppe}{Gruppe 2 -- Florian Seidler}
\newcommand{\theuebung}{3}

\ifoot{\footnotesize{Martin Ueding}}
\ihead{\themodul{} -- Übung \theuebung}
\ofoot{\footnotesize{\thegruppe}}

\def\thesubsection{\thesection\alph{subsection}}

\title{\themodul{} -- Übung \theuebung}
\subtitle{\thegruppe}
\author{
	Martin Ueding \footnote{\href{mailto:mu@uni-bonn.de}{mu@uni-bonn.de}}
}

\hypersetup{
	pdftitle={\themodul {} - Übung \theuebung},
}

\begin{document}

\maketitle

\begin{center}
	\ccbysadetitle
\end{center}

\begin{table}[h]
	\centering
	\begin{tabular}{l|c|c|c|c|c}
		Aufgabe
		& \ref 1
		& \ref 2
		& \ref 3
		& \ref 4
		& $\sum$   \\
		\hline
		Punkte
		& \punkte / 4
		& \punkte / 8
		& \punkte / 8
		& \punkte / 8
		& \punkte / 28
	\end{tabular}
\end{table}

%%%%%%%%%%%%%%%%%%%%%%%%%%%%%%%%%%%%%%%%%%%%%%%%%%%%%%%%%%%%%%%%%%%%%%%%%%%%%%%
%                                 Photoeffekt                                 %
%%%%%%%%%%%%%%%%%%%%%%%%%%%%%%%%%%%%%%%%%%%%%%%%%%%%%%%%%%%%%%%%%%%%%%%%%%%%%%%

\section{Photoeffekt}
\label 1

\subsection{Grenzwellenlänge}

\newcommand\UA{U_\text{A}}

Es muss $\hbar \omega \geq \UA$ gelten. Nach $f$ umgeformt bleibt:
\[
	f \geq \frac\UA h = \SI{5.08e14}\hertz
\]

Und die dazugehörige Wellenlänge:
\[
	\lambda \leq \frac{ch}\UA = \SI{5.90e-7}\meter
\]

\subsection{Geschwindigkeit}

Ich forme nach $v$ um:
\[
	h \frac c\lambda - \UA = \half m_\text{e} v^2
	\implies
	v = \SI{712000}{\meter\per\second}
\]

%%%%%%%%%%%%%%%%%%%%%%%%%%%%%%%%%%%%%%%%%%%%%%%%%%%%%%%%%%%%%%%%%%%%%%%%%%%%%%%
%                                    Ende                                     %
%%%%%%%%%%%%%%%%%%%%%%%%%%%%%%%%%%%%%%%%%%%%%%%%%%%%%%%%%%%%%%%%%%%%%%%%%%%%%%%

\IfFileExists{\bibliographyfile}{
	%\bibliography{\bibliographyfile}
}{}

\end{document}

% vim: spell spelllang=de
