% Copyright © 2013 Martin Ueding <dev@martin-ueding.de>
%
\input{header.tex}

\usepackage{tikz}
\usetikzlibrary{calc}

\newcommand{\themodul}{physik411}
\newcommand{\thegruppe}{Gruppe 2 -- Florian Seidler}
\newcommand{\theuebung}{6}

\ifoot{\footnotesize{Martin Ueding}}
\ihead{\themodul{} -- Übung \theuebung}
\ofoot{\footnotesize{\thegruppe}}

\def\thesubsection{\thesection\alph{subsection}}

\title{\themodul{} -- Übung \theuebung}
\subtitle{\thegruppe}
\author{
	Martin Ueding \footnote{\href{mailto:mu@uni-bonn.de}{mu@uni-bonn.de}}
}

\hypersetup{
	pdftitle={\themodul {} - Übung \theuebung},
}

\begin{document}

\maketitle

\begin{center}
	\ccbysadetitle
\end{center}

\begin{Form}
	\begin{table}[h]
		\centering
		\begin{tabular}{l|c|c|c|c|c|c|c}
			Aufgabe
			& \ref 1
			& \ref 2
			& \ref 3
			& \ref 4
			& \ref 5
			& \ref 6
			& $\sum$   \\
			\hline
			Punkte
			& \TextField[name=aufgabe1, width=1cm]{} / 3
			& \TextField[name=aufgabe2, width=1cm]{} / 4
			& \TextField[name=aufgabe3, width=1cm]{} / 5
			& \TextField[name=aufgabe4, width=1cm]{} / 4
			& \TextField[name=aufgabe5, width=1cm]{} / 7
			& \TextField[name=aufgabe6, width=1cm]{} / 5
			& \TextField[name=ergebnis, width=1cm]{} / 28
		\end{tabular}
	\end{table}
\end{Form}

%%%%%%%%%%%%%%%%%%%%%%%%%%%%%%%%%%%%%%%%%%%%%%%%%%%%%%%%%%%%%%%%%%%%%%%%%%%%%%%
%                                 Präzession                                 %
%%%%%%%%%%%%%%%%%%%%%%%%%%%%%%%%%%%%%%%%%%%%%%%%%%%%%%%%%%%%%%%%%%%%%%%%%%%%%%%

\section{Präzession}
\label 1

Mit der Larmorfrequenz
\[
	\omega_\text L = \frac{g_\text L \mu_\text B B_z}\hbar
\]
und den gegeben Werten erhalte ich $\omega_\text L =
\SI{4.39515e6}{\radian\per\second}$. Mit den anderen Werten erhalte ich
$\omega_\text L = \SI{13405.7}{\radian\per\second}$.

%%%%%%%%%%%%%%%%%%%%%%%%%%%%%%%%%%%%%%%%%%%%%%%%%%%%%%%%%%%%%%%%%%%%%%%%%%%%%%%
%                             Paschen-Back-Effekt                             %
%%%%%%%%%%%%%%%%%%%%%%%%%%%%%%%%%%%%%%%%%%%%%%%%%%%%%%%%%%%%%%%%%%%%%%%%%%%%%%%

\section{Paschen-Back-Effekt}
\label 2

3p nach 3s bedeutet, dass $n = 3$ und $l$ von 2 auf 1 wechselt.

Der Zeeman-Effekt tritt bei $L, S = 0$ auf, also bei zwei-Elektronen-Atomen.
Der anomale Zeeman-Effekt tritt bei $L,S > 0$ auf, also bei
ein-Elektronen-Atomen. In der gleichen Konfiguration tritt auch der
Paschen-Back-Effekt auf. [Vorlesung 10, Folie 1]

Bei einem externen Magnetfeld $B$, dass deutlich kleiner als das Magnetfeld der
Spin-Bahn-Kopplung $B_\text{SO}$ ist, tritt der anormale Zeeman-Effekt auf. Bei
größeren Magnetfeldern ist der Übergang zum Paschen-Back-Effekt.

\fehlt

%%%%%%%%%%%%%%%%%%%%%%%%%%%%%%%%%%%%%%%%%%%%%%%%%%%%%%%%%%%%%%%%%%%%%%%%%%%%%%%
%                          Spektroskopische Notation                          %
%%%%%%%%%%%%%%%%%%%%%%%%%%%%%%%%%%%%%%%%%%%%%%%%%%%%%%%%%%%%%%%%%%%%%%%%%%%%%%%

\section{Spektroskopische Notation}
\label 3

\subsection{In Notation schreiben}

\[
	n = 3
	\eqnsep
	l = 3
	\eqnsep
	j = \frac 52
	\iff
	3 \mathrm D_{5/2}
\]
\[
	n = 2
	\eqnsep
	l = 2
	\eqnsep
	j = \frac 32
	\iff
	2 \mathrm P_{3/2}
\]

\subsection{Notation lesen}

\[
	6^2 \mathrm P_{3/2}
	\iff
	n = 6
	\eqnsep
	l = 2
	\eqnsep
	s = \frac 12
	\eqnsep
	j = \frac 32
\]
\[
	6^2 \mathrm S_{1/2}
	\iff
	n = 6
	\eqnsep
	l = 1
	\eqnsep
	s = \half
	\eqnsep
	j = \half
\]

%%%%%%%%%%%%%%%%%%%%%%%%%%%%%%%%%%%%%%%%%%%%%%%%%%%%%%%%%%%%%%%%%%%%%%%%%%%%%%%
%                                Auswahlregeln                                %
%%%%%%%%%%%%%%%%%%%%%%%%%%%%%%%%%%%%%%%%%%%%%%%%%%%%%%%%%%%%%%%%%%%%%%%%%%%%%%%

\section{Auswahlregeln}
\label 4

Die Auswahlregeln sind: \cite[Tabelle 15.8]{meschede-gerthsen_24}
\begin{gather*}
	\Deltaup l = \pm 1 \\
	\Deltaup m = 0, \pm 1 \\
	\Deltaup s = 0 \\
	\Deltaup j = 0, \pm 1 \\
	\Deltaup m_j = 0, \pm 1
\end{gather*}

Außerdem darf nicht $j\colon 0 \to 0$ sowie $\Deltaup m_j = 0$ bei $\Deltaup j
= 0$ stattfinden.

Die erlaubten Übergänge sind in Abbildung \ref{fig:termschema}.

\begin{figure}
	\centering
	\includegraphics{4-Termschema-bearbeitet.pdf}
	\caption{Ausgefülltes Termschema}
	\label{fig:termschema}
\end{figure}

%%%%%%%%%%%%%%%%%%%%%%%%%%%%%%%%%%%%%%%%%%%%%%%%%%%%%%%%%%%%%%%%%%%%%%%%%%%%%%%
%                                Quantendefekt                                %
%%%%%%%%%%%%%%%%%%%%%%%%%%%%%%%%%%%%%%%%%%%%%%%%%%%%%%%%%%%%%%%%%%%%%%%%%%%%%%%

\section{Quantendefekt}
\label 5

\subsection{Energien und Wellenlängen}

\begin{tabular}{ccS}
	{$n$} & {$l$} & {$E_{n,l}/\si\electronvolt$} \\
	\hline
	1 & s & 944.84 \\
	1 & p & 99.3842 \\
	1 & d & 13.8819 \\
	2 & s & 10.8464 \\
	2 & p & 34.2799 \\
	2 & d & 3.43569 \\
	3 & s & 3.02725 \\
	3 & p & 5.12089 \\
	3 & d & 1.52187 \\
\end{tabular}

Die Natrium-D-Linie ist der Übergang von 3p nach 3s. Die Wellenlänge dazu ist
$\lambda = \SI{592.196}{\nano\meter}$.

\subsection{Kernladung, Radialwellenfunktion}

Die äußeren Elektronen schirmen die Kernladung ab. Wäre das Natrium nicht
neutral, sondern fast vollständig ionisiert, müsste das $Z$ mit in die Formel
rein. Die Abschirmung steckt im Quantendefekt drin.

Die verschiedenen Orbitale unterscheiden sich für die s-Orbitale nur im
Radialteil, für p und vor allem d kommen deutlich andere Winkelfunktionen.
Außerdem haben die assoziierten Laguerre-Polynome, von dem was ich aus dem
Bildern der Vorlesung ablesen kann, $n - l-1$ Nullstellen, so dass mit höherem
$l$ die Funktion weniger stark vom Radius abhängt.  Daher sollten gerade die
s-Orbitale empfindlich in ihrer Energie gestört werden.

%%%%%%%%%%%%%%%%%%%%%%%%%%%%%%%%%%%%%%%%%%%%%%%%%%%%%%%%%%%%%%%%%%%%%%%%%%%%%%%
%                                 Lorenzkurve                                 %
%%%%%%%%%%%%%%%%%%%%%%%%%%%%%%%%%%%%%%%%%%%%%%%%%%%%%%%%%%%%%%%%%%%%%%%%%%%%%%%

\section{Lorenzkurve}
\label 6

\subsection{Fouriertransformation}

Ich berechne die Fouriertransformation zu der gegebenen Entwicklung des
elektrischen Feldes $E$:
\begin{align*}
	E(\omega)
	&= \int_0^\infty \dif t \, E(t) \exp\del{-\ii \omega t} \\
	&= \int_0^\infty \dif t \, E_0 \exp\del{\ii \del{\frac{\ii \Gamma}2 + \omega_0 - \omega} t} \\
	&= E_0 \frac1{-\frac \Gamma2 + \ii \del{\omega_0 - \omega}}
\end{align*}

\subsection{Intensitätsspektrum}

\begin{align*}
	P(\omega)
	&= \abs{E(\omega)}^2 \\
	&= E_0^2 \frac1{\frac{\Gamma^2}4 + \del{\omega_0 - \omega}^2}
\end{align*}

Die Maximale Intensität ist bei $\omega = \omega_0$. Die halbe Intensität ergibt sich bei
\[
	\del{\omega - \omega_0}^2 = \frac{\Gamma^2}4,
\]

Also bei $\abs{\omega_0 - \omega} = \Gamma / \sqrt 2$. Die volle Halbwertsbreite ist somit:
\[
	\Deltaup \omega_\text{FWHM} = \sqrt 2 \Gamma
\]

\subsection{Cäsium}

Bei Cäsium ist $\tau = \SI{30}{\nano\second}$, $\Gamma = 1/\tau$.
\[
	\Deltaup\nu = \frac{\piup}{\sqrt 2 \tau} = \SI{7.40e7}\hertz
\]

%%%%%%%%%%%%%%%%%%%%%%%%%%%%%%%%%%%%%%%%%%%%%%%%%%%%%%%%%%%%%%%%%%%%%%%%%%%%%%%
%                                    Ende                                     %
%%%%%%%%%%%%%%%%%%%%%%%%%%%%%%%%%%%%%%%%%%%%%%%%%%%%%%%%%%%%%%%%%%%%%%%%%%%%%%%

\IfFileExists{\bibliographyfile}{
	\bibliography{\bibliographyfile}
}{}

\end{document}

% vim: spell spelllang=de
