% Copyright © 2013 Martin Ueding <dev@martin-ueding.de>
%
\input{header.tex}

\usepackage{pdfpages}
\usepackage{tikz}
\usetikzlibrary{calc}

\newcommand{\themodul}{physik411}
\newcommand{\thegruppe}{Gruppe 2 -- Florian Seidler}
\newcommand{\theuebung}{10}

\ifoot{\footnotesize{Martin Ueding}}
\ihead{\themodul{} -- Übung \theuebung}
\ofoot{\footnotesize{\thegruppe}}

\def\thesubsection{\thesection\alph{subsection}}

\title{\themodul{} -- Übung \theuebung}
\subtitle{\thegruppe}
\author{
	Martin Ueding \footnote{\href{mailto:mu@uni-bonn.de}{mu@uni-bonn.de}}
}

\hypersetup{
	pdftitle={\themodul {} - Übung \theuebung},
}

\begin{document}

\maketitle

\begin{center}
	\ccbysadetitle
\end{center}

\begin{table}[h]
	\centering
	\begin{tabular}{l*4{|c}}
		Aufgabe
		& \ref 1
		& \ref 2
		& \ref 3
		& $\sum$   \\
		\hline
		Punkte
		& \punkte / 14
		& \punkte / 10
		& \punkte / 16
		& \punkte / 40
	\end{tabular}
\end{table}

%%%%%%%%%%%%%%%%%%%%%%%%%%%%%%%%%%%%%%%%%%%%%%%%%%%%%%%%%%%%%%%%%%%%%%%%%%%%%%%
%                             Die Brillouin-Zone                              %
%%%%%%%%%%%%%%%%%%%%%%%%%%%%%%%%%%%%%%%%%%%%%%%%%%%%%%%%%%%%%%%%%%%%%%%%%%%%%%%

\section{Die Brillouin-Zone}
\label 1

\subsection{Definition}

Die erste Brillouin-Zone ist die primitive Wiener-Seitz-Zelle im reziproken Raum.
\cite{wikipedia/Brillouin-Zone}

\subsection{Konstruktion}

Im reziproken Gitter:

\begin{itemize}
	\item Punkt aussuchen
	\item Verbindung zu nächsten Nachbarn einzeichnen
	\item Mittelsenkrechten in die Verbindungen einzeichnen.
	\item Umschlossene Fläche ist erste Zone.
\end{itemize}

\subsection{Symmetrie}

\begin{small}
Die transnationale Symmetrie ist eine andere Formulierung des Mach'schen
Prinzips. Wenn das Universum überall gleich ist, dann sind es auch Kristalle.
Und wenn das im ganzen Universum gilt, dann auch in den $\approx \num{e2}$
Ländern auf der Erde.
\end{small}

Die Bragg-Reflexion tritt immer auf, wenn $\Deltaup \vec k = \vec g$, also ein
beliebiger reziproker Gittervektor, ist. Der Bloch-Satz besagt, dass in der
Welle ein Phasenfaktor auftritt: \cite[Vorlesung 16, Folie
7]{meschede/physik441}
\[
	\exp\del{\ii \inner{\vec k}{\vec t_n}}
	\eqnsep
	\inner{\vec t_n}{\vec g_m} = 2\piup n
\]

Wenn also eine Transformation $\vec k' := \vec k + \vec g$ durchgeführt wird,
ändert sich die Phase gerade nicht. Somit sind die Zonen zueinander äquivalent.

\subsection{Äquivalente Zonen}

Die Zonen habe ich per Hand in die gegebene Abbildung gemalt, siehe nächste
Seite.

Dabei habe ich jedoch nur $6+6+6=18$ Maschen eingezeichnet. Die Sechsecke sind
allerdings komplett belegt. Wo ist da der Fehler?

\includepdf[pages=2]{gitter.pdf}

%%%%%%%%%%%%%%%%%%%%%%%%%%%%%%%%%%%%%%%%%%%%%%%%%%%%%%%%%%%%%%%%%%%%%%%%%%%%%%%
%                                    Ende                                     %
%%%%%%%%%%%%%%%%%%%%%%%%%%%%%%%%%%%%%%%%%%%%%%%%%%%%%%%%%%%%%%%%%%%%%%%%%%%%%%%

\IfFileExists{\bibliographyfile}{
	\bibliography{\bibliographyfile}
}{}

\end{document}

% vim: spell spelllang=de
