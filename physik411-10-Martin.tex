% Copyright © 2013 Martin Ueding <dev@martin-ueding.de>
%
\input{header.tex}

\usepackage{pdfpages}
\usepackage{tikz}
\usepackage{cleveref}
\usetikzlibrary{calc}

\newcommand{\themodul}{physik411}
\newcommand{\thegruppe}{Gruppe 2 -- Florian Seidler}
\newcommand{\theuebung}{10}

\ifoot{\footnotesize{Martin Ueding}}
\ihead{\themodul{} -- Übung \theuebung}
\ofoot{\footnotesize{\thegruppe}}

\def\thesubsection{\thesection\alph{subsection}}

\title{\themodul{} -- Übung \theuebung}
\subtitle{\thegruppe}
\author{
	Martin Ueding \footnote{\href{mailto:mu@uni-bonn.de}{mu@uni-bonn.de}}
}

\hypersetup{
	pdftitle={\themodul {} - Übung \theuebung},
}

\begin{document}

\maketitle

\begin{center}
	\ccbysadetitle
\end{center}

\begin{table}[h]
	\centering
	\begin{tabular}{l*4{|c}}
		Aufgabe
		& \ref 1
		& \ref 2
		& \ref 3
		& $\sum$   \\
		\hline
		Punkte
		& \punkte / 14
		& \punkte / 10
		& \punkte / 16
		& \punkte / 40
	\end{tabular}
\end{table}

%%%%%%%%%%%%%%%%%%%%%%%%%%%%%%%%%%%%%%%%%%%%%%%%%%%%%%%%%%%%%%%%%%%%%%%%%%%%%%%
%                             Die Brillouin-Zone                              %
%%%%%%%%%%%%%%%%%%%%%%%%%%%%%%%%%%%%%%%%%%%%%%%%%%%%%%%%%%%%%%%%%%%%%%%%%%%%%%%

\section{Die Brillouin-Zone}
\label 1

\subsection{Definition}

Die erste Brillouin-Zone ist die primitive Wiener-Seitz-Zelle im reziproken Raum.
\cite{wikipedia/Brillouin-Zone}

\subsection{Konstruktion}

Im reziproken Gitter:

\begin{itemize}
	\item Punkt aussuchen
	\item Verbindung zu nächsten Nachbarn einzeichnen
	\item Mittelsenkrechten in die Verbindungen einzeichnen.
	\item Umschlossene Fläche ist erste Zone.
\end{itemize}

\subsection{Symmetrie}

Die Bragg-Reflexion tritt immer auf, wenn $\Deltaup \vec k = \vec g$, also ein
beliebiger reziproker Gittervektor, ist. Der Bloch-Satz besagt, dass in der
Welle ein Phasenfaktor auftritt: \cite[Vorlesung~16,
Folie~7]{meschede/physik441}
\[
	\exp\del{\ii \inner{\vec k}{\vec t_n}}
	\eqnsep
	\inner{\vec t_n}{\vec g_m} = 2\piup n
\]

Wenn also eine Transformation $\vec k' := \vec k + \vec g$ durchgeführt wird,
ändert sich die Phase gerade nicht. Somit sind die Zonen zueinander äquivalent.

\subsection{Äquivalente Zonen}

Die Zonen habe ich per Hand in die gegebene Abbildung gemalt, siehe nächste
Seite.

Dabei habe ich jedoch nur $6+6+6=18$ Maschen eingezeichnet. Die Sechsecke sind
allerdings komplett belegt. Wo ist da der Fehler?

\includepdf[pages=2]{gitter.pdf}

%%%%%%%%%%%%%%%%%%%%%%%%%%%%%%%%%%%%%%%%%%%%%%%%%%%%%%%%%%%%%%%%%%%%%%%%%%%%%%%
%                                 Edelmetalle                                 %
%%%%%%%%%%%%%%%%%%%%%%%%%%%%%%%%%%%%%%%%%%%%%%%%%%%%%%%%%%%%%%%%%%%%%%%%%%%%%%%

\section{Edelmetalle}
\label 2

\subsection{Bravaisgitter}

Kupfer ist „cubic-close-packed“, also ccp.
\cite{webelements/copper/crystal_structure}

\subsection{Erste 3D Zone}

\fehlt

\cite{akopian/brillouin-zones}
\cite{chang/intro_solid_state}

\subsection{Hybridisierungstypen}

\fehlt

%%%%%%%%%%%%%%%%%%%%%%%%%%%%%%%%%%%%%%%%%%%%%%%%%%%%%%%%%%%%%%%%%%%%%%%%%%%%%%%
%                  Optische Gitter und Tight-Binding-Methode                  %
%%%%%%%%%%%%%%%%%%%%%%%%%%%%%%%%%%%%%%%%%%%%%%%%%%%%%%%%%%%%%%%%%%%%%%%%%%%%%%%

\section{Optische Gitter und Tight-Binding-Methode}
\label 3

\subsection{Oszillatorfrequenz}

Gegeben ist das Potential:
\[
	\hat V(x) = \frac{U_0}2 \del{1-\cos\del{\frac{2\piup}a x}}
\]

Ein Minimum ist bei $x = 0$. Darum entwickele ich die Funktion und erhalte:
\[
	\hat V(x) = U_0 \frac{\piup^2}{a^2} x^2 + \mathcal O(x^3)
\]

Ein Koeffizientenvergleich mit $U(x) = m \omega^2 x^2/2$ liefert:
\[
	\omega^2 = 2 U_0 \frac{\piup^2}{a^2m}
\]

\subsection{Bedingung an Potentialtiefe}

Das $\Deltaup x$ sieht nach dem FWHM für die Wellenfunktion des Zustandes $\ket 0$ des harmonischen Oszillators aus. Nach \cite[Übersichtsfolie 17]{kubis/physik421} ist das:
\[
	\psi_0(x) = \del{\frac{m \omega}{\piup \hbar}}^{\frac 14} \exp\del{-\frac{m\omega}{2\hbar} x^2}
\]

Die halbe Höhe ist erreicht, wenn:
\[
	x = \pm \sqrt{\frac{2\hbar \ln(2)}{m\omega}}
\]

Somit ist
\[
	\Deltaup x = 2 \sqrt{\frac{2\hbar \ln(2)}{m\omega}}
\]

Dies soll jetzt deutlich kleiner als $a$ sein. Nach drei Umformungen und dem
Einsetzen des gefundenen $\omega$ erhalte ich:
\[
	32 \frac{\hbar^2 \ln^2(2) a^2 m}{\piup^2} \ll U_0
\]

\subsection{Diagonalisieren}

Gegeben ist:
\[
	\hat H = \begin{pmatrix}
		E & J \\ J & E
	\end{pmatrix}
\]

Ich suche die Eigenvektoren, diese sind:
\[
	\vec e_1 = \begin{pmatrix}
		-1 \\ 1
	\end{pmatrix}
	\eqnsep
	\vec e_2 = \begin{pmatrix}
		1 \\ 1
	\end{pmatrix}
\]

Somit kann ich die Eigenzustände schreiben als:
\[
	\ket{e_1} = \frac{1}{\sqrt{2}} \vec e_2 \\
	\ket{e_2} = \frac{1}{\sqrt{2}} \vec e_1 \\
\]

\subsection{Skizze}

Die Skizze der komplizierten Formel, deren Herleitung gar nicht so wichtig ist,
ist in \cref{fig:J}.

\begin{figure}
	\centering
	\includegraphics[width=\linewidth]{3d.pdf}
	\caption{}
	\label{fig:J}
\end{figure}

\subsection{Tunnelrate durch Barriere}

\fehlt

\subsection{Translationsinvariate Matrixform}

Die Funktion $\psi_k$ erfüllt das Bloch-Theorem. Der reziproke
Gitterbasisvektor ist $g = 2\piup/a$. Eine Translation im $k$-Raum um $k
\mapsto k + g$ führt in der Exponentialfunktion zur Addition von $2\piup \ii$,
so dass nichts verändert wird.

$\psi_0$ ist die Wellenfunktion vom Oszillator an der Position 0. Die Funktion $\psi_k$ kann in der Basis der $\psi_n$ dargestellt werden als:
\[
	\psi_k =
	\begin{pmatrix}
		\vdots \\
		\exp(-2 \ii k a) \\
		\exp(- \ii k a) \\
		1 \\
		\exp(\ii k a) \\
		\exp(2 \ii k a) \\
		\vdots \\
	\end{pmatrix}
\]

Wenn ich nun den Hamilonoperator darauf anwende, erhalte ich:
\[
	\begin{pmatrix}
		\vdots \\
		J \exp(-2\ii ka) + E_0 \exp(-\ii ka) + J \\
		J \exp(-\ii ka) + E_0 + J \exp(\ii ka) \\
		J + E_0 \exp(\ii ka) + J \exp(2\ii ka) \\
		\vdots \\
	\end{pmatrix}
	=
	\del{J \exp(\ii ka) + E_0 + J \exp(-\ii ka)}
	\begin{pmatrix}
		\vdots \\
		\exp(-2 \ii k a) \\
		\exp(- \ii k a) \\
		1 \\
		\exp(\ii k a) \\
		\exp(2 \ii k a) \\
		\vdots \\
	\end{pmatrix}
\]

Somit ist der Energieeigenwert:
\[
	J \exp(\ii ka) + E_0 + J \exp(-\ii ka)
	=
	E_0 + 2 J \cos(ka)
\]

\subsection{Effektive Masse}

Gegeben ist in der vorherigen Aufgabe:
\[
	E = E_0 + 2 J \cos(ka)
\]

Dies leite ich zweimal nach $k$ ab, invertiere es und erhalte:
\[
	m_\text{eff} = - \frac{\hbar}{2Ja^2}
\]

Da $J$ negativ ist (siehe \cref{fig:J}), ist die Masse auch positiv.

%%%%%%%%%%%%%%%%%%%%%%%%%%%%%%%%%%%%%%%%%%%%%%%%%%%%%%%%%%%%%%%%%%%%%%%%%%%%%%%
%                                    Ende                                     %
%%%%%%%%%%%%%%%%%%%%%%%%%%%%%%%%%%%%%%%%%%%%%%%%%%%%%%%%%%%%%%%%%%%%%%%%%%%%%%%

\IfFileExists{\bibliographyfile}{
	\bibliography{\bibliographyfile}
}{}

\end{document}

% vim: spell spelllang=de
