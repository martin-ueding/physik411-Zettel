% Copyright © 2012-2013 Martin Ueding <dev@martin-ueding.de>
%
\input{header.tex}

\usepackage{tikz}

\newcommand{\themodul}{physik411}
\newcommand{\thegruppe}{Gruppe 2}
\newcommand{\theuebung}{1}

\ifoot{\footnotesize{Martin Ueding}}
\ihead{\themodul{} -- Übung \theuebung}
\ofoot{\footnotesize{\thegruppe}}

\def\thesubsection{\thesection\alph{subsection}}

\title{\themodul{} -- Übung \theuebung \\ \vspace{0.5cm} \large{\thegruppe}}

\author{
	Martin Ueding \\ \small{\href{mailto:mu@uni-bonn.de}{mu@uni-bonn.de}}
}

\hypersetup{
	pdftitle={\themodul {} - Übung \theuebung},
}

\begin{document}

\maketitle

\begin{table}[h]
	\centering
	\begin{tabular}{l|c|c|c|c|c|c}
		Aufgabe
		& \ref 1
		& \ref 2
		& \ref 3
		& \ref 4
		& \ref 5
		& $\sum$   \\
		\hline
		Punkte
		& \punkte / 8
		& \punkte / 12
		& \punkte / 5
		& \punkte / 9
		& \punkte / 9
		& \punkte / 43
	\end{tabular}
\end{table}


\section{Moleküle zählen}
\label 1

\section{Flugzeit-Massenspektrometer}
\label 2

\section{Mittlere freie Weglänge}
\label 3

\section{De Broglie-Wellenlängen}
\label 4

\section{Kastenpotential}
\label 5

%%%%%%%%%%%%%%%%%%%%%%%%%%%%%%%%%%%%%%%%%%%%%%%%%%%%%%%%%%%%%%%%%%%%%%%%%%%%%%%
%                                    Ende                                     %
%%%%%%%%%%%%%%%%%%%%%%%%%%%%%%%%%%%%%%%%%%%%%%%%%%%%%%%%%%%%%%%%%%%%%%%%%%%%%%%

%\IfFileExists{\bibliographyfile}{
	%\bibliography{\bibliographyfile}
	%\bibliographystyle{plain}
%}{}

\end{document}

% vim: spell spelllang=de
