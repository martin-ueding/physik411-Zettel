% Copyright © 2013 Martin Ueding <dev@martin-ueding.de>
%
\input{header.tex}

\usepackage{tikz}

\newcommand{\themodul}{physik411}
\newcommand{\thegruppe}{Gruppe 2}
\newcommand{\theuebung}{1}

\ifoot{\footnotesize{Martin Ueding}}
\ihead{\themodul{} -- Übung \theuebung}
\ofoot{\footnotesize{\thegruppe}}

\def\thesubsection{\thesection\alph{subsection}}

\title{\themodul{} -- Übung \theuebung \\ \vspace{0.5cm} \large{\thegruppe}}

\author{
	Martin Ueding \\ \small{\href{mailto:mu@uni-bonn.de}{mu@uni-bonn.de}}
}

\hypersetup{
	pdftitle={\themodul {} - Übung \theuebung},
}

\begin{document}

\maketitle

\begin{small}
	Symbolerklärung: Nabla $\vnabla$, Laplace $\laplace$, d'Alambert
	$\dalambert$, Delta $\Delta$ und $\Deltaup$, Vektor $\vec v$, Tensor $\tens
	T$.
\end{small}

\begin{table}[h]
	\centering
	\begin{tabular}{l|c|c|c|c|c|c}
		Aufgabe
		& \ref 1
		& \ref 2
		& \ref 3
		& \ref 4
		& \ref 5
		& $\sum$   \\
		\hline
		Punkte
		& \punkte / 8
		& \punkte / 12
		& \punkte / 5
		& \punkte / 9
		& \punkte / 9
		& \punkte / 43
	\end{tabular}
\end{table}

\section{Moleküle zählen}
\label 1

\subsection{Wasserglas}

\SI{1}{mol} Wasser wiegt \SI{18}{g}. Dort drin sind \num{6.02e23} Moleküle
enthalten. Das Volumen ist \SI{18}{ml}. Im ganzen Glas sind also \num{6.69e24}
Moleküle enthalten. Da \SI{0.2}{l} nur eine signifikante Stelle hat, ist meine
Antwort \num{7e24} Moleküle.

\subsection{Markierung}

Zuerst schätze ich ab, welche Wassermenge der Planet hat. Dazu nehme ich an, dass, wenn man die Meere gleichmäßig verteilt, sie eine Tiefe von vielleicht \SI{1000}{m} haben. Das Volumen ist dann:
\[
	V = 4 \piup R_\text{E}^2 \cdot \SI{1000}{m}
	= \SI{5.15e17}{m^3}
\]

Geteilt durch die \SI{18}{ml/mol} erhalte ich eine Stoffmenge von
\SI{2.86e19}{mol}, was \num{1.72e43} Teilchen entspricht.

Im Meer ist jetzt ein Anteil von \num{3.00e-26} Markiert. Wenn ich jetzt wieder
\num{6.69e24} Moleküle auswähle, dann ist der Erwartungswert \num{1.54e-8}
Moleküle, also werden keine markierten in dem Glas sein.

\section{Flugzeit-Massenspektrometer}
\label 2

\subsection{Skizze}

\begin{center}
	\begin{tikzpicture}
		\draw[dashed] (0, 0) -- ++(0, 3);
		\draw[dashed] (2, 0) -- ++(0, 3);
		\node[below] at (1, 0) {Kondensatorplatten};
		\draw[|-|] (0, 3.2) -- ++(2, 0) node[above, midway] {\SI{10}{mm}};
		\node[draw, rectangle, right] at (10, 1.5) {Detektor};
		\draw[|-|] (2, 3.2) -- (10, 3.2) node[above, midway] {\SI{1}{m}};
		\node[draw, circle] at (1, 1.5) {$\mathrm{NO}$};
	\end{tikzpicture}
\end{center}

\subsection{Flugzeit für $\mathrm{^{14}N^{16}O}$}

Die Masse des Isotops ist $\SI{30}{amu} = \SI{4.98e-23}{kg}$. Das Ion legt eine
Potentialdifferenz von \SI{500}{V} zurück (mittig im Kondensator), es bekommt
also $\SI{500}{eV} = \SI{8.05e-17}{J}$ Energie. Mit $E_\text{kin} = mv^2/2$
erhalte ich eine Geschwindigkeit von $v = \SI{56800}{m/s}$. Da die Ruhemasse im
Bereich von \si{MeV} liegt, darf ich an dieser Stelle klassisch rechnen. Mit
dieser Geschwindigkeit $v$ legen die Ionen die Strecke von $L = \SI1m$ in
$t = \SI{1.76e-5}s$ zurück.

Dabei habe ich die Beschleunigungszeit nicht berücksichtigt. Das Ion
beschleunigt innerhalb von $d/2 = \SI5{mm}$ auf die Geschwindigkeit $v$. Mit
$v^2 = 2ad/2$ erhalte ich eine Beschleunigung von $a = \SI{3.23e11}{m/s^2}$.
Die Zeit, um diese Strecke zurück zu legen erhalte ich mit $d/2 = a \tilde
t^2/2$: $\tilde t = \SI{1.76e-7}s$. Der relative Fehler ist also \num{0.01}.

\subsection{Flugzeitverbreiterung}

Für die Flugzeitverbreiterung betrachte ich zwei Ionen, die an beiden Enden des Wechselwirkungsgebietes gestartet sind. Die Ionen bekommen nun als Energie:
\[
	E_\pm = \frac{\frac d2 \pm \frac{\Deltaup x}2}d e U_0
\]

Diese Energien sind:
\[
	E_+ = \SI{8.13e-17}J
	\eqnsep
	E_- = \SI{7.96e-17}J
\]

Mit den eben benutzten Formeln errechne ich die Flugzeit für beide Ionen aus:
\[
	t_+ = \SI{1.75056e-5}s
	\eqnsep
	t_- = \SI{1.76815e-5}s
\]

Die Zeitauflösung ist also maximal \SI{1.759e-7}s, wenn die endliche Größe des
Wechselwirkungsgebietes berücksichtigt wird. Nun berechne ich die Flugzeit für die verschiedenen Isotope, wenn sie aus der Mitte des Kondensators starten:
\begin{center}
	\begin{tabular}{lSSS}
		Isotop & {Masse / \si{amu}} & {Flugzeit $t$ / \si s} \\
		\hline
		$\mathrm{^{14}N^{16}O}$ & 30 & 1.75929e-5 \\
		$\mathrm{^{15}N^{16}O}$ & 31 & 1.78837e-5 \\
		$\mathrm{^{14}N^{18}O}$ & 32 & 1.81689e-5 \\
	\end{tabular}
\end{center}

Die Werte unterschieden sich mehr als \SI{2e-7}s, so dass eine Unterscheidung
möglich ist, wenn auch knapp.

\section{Mittlere freie Weglänge}
\label 3

\section{De Broglie-Wellenlängen}
\label 4

\section{Kastenpotential}
\label 5

%%%%%%%%%%%%%%%%%%%%%%%%%%%%%%%%%%%%%%%%%%%%%%%%%%%%%%%%%%%%%%%%%%%%%%%%%%%%%%%
%                                    Ende                                     %
%%%%%%%%%%%%%%%%%%%%%%%%%%%%%%%%%%%%%%%%%%%%%%%%%%%%%%%%%%%%%%%%%%%%%%%%%%%%%%%

%\IfFileExists{\bibliographyfile}{
	%\bibliography{\bibliographyfile}
	%\bibliographystyle{plain}
%}{}

\end{document}

% vim: spell spelllang=de
