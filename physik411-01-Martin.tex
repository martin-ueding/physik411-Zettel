% Copyright © 2013 Martin Ueding <dev@martin-ueding.de>
%
\input{header.tex}

\usepackage{tikz}

\newcommand{\themodul}{physik411}
\newcommand{\thegruppe}{Gruppe 2}
\newcommand{\theuebung}{1}

\ifoot{\footnotesize{Martin Ueding}}
\ihead{\themodul{} -- Übung \theuebung}
\ofoot{\footnotesize{\thegruppe}}

\def\thesubsection{\thesection\alph{subsection}}

\title{\themodul{} -- Übung \theuebung \\ \vspace{0.5cm} \large{\thegruppe}}

\author{
	Martin Ueding \\ \small{\href{mailto:mu@uni-bonn.de}{mu@uni-bonn.de}}
}

\hypersetup{
	pdftitle={\themodul {} - Übung \theuebung},
}

\begin{document}

\maketitle

\begin{small}
	Symbolerklärung: Nabla $\vnabla$, Laplace $\laplace$, d'Alambert
	$\dalambert$, Delta $\Delta$ und $\Deltaup$, Vektor $\vec v$, Tensor $\tens
	T$.
\end{small}

\begin{table}[h]
	\centering
	\begin{tabular}{l|c|c|c|c|c|c}
		Aufgabe
		& \ref 1
		& \ref 2
		& \ref 3
		& \ref 4
		& \ref 5
		& $\sum$   \\
		\hline
		Punkte
		& \punkte / 8
		& \punkte / 12
		& \punkte / 5
		& \punkte / 9
		& \punkte / 9
		& \punkte / 43
	\end{tabular}
\end{table}

\section{Moleküle zählen}
\label 1

\subsection{Wasserglas}

\SI{1}{mol} Wasser wiegt \SI{18}{g}. Dort drin sind \num{6.02e23} Moleküle
enthalten. Das Volumen ist \SI{18}{ml}. Im ganzen Glas sind also \num{6.69e24}
Moleküle enthalten. Da \SI{0.2}{l} nur eine signifikante Stelle hat, ist meine
Antwort \num{7e24} Moleküle.

\subsection{Markierung}

Zuerst schätze ich ab, welche Wassermenge der Planet hat. Dazu nehme ich an,
dass, wenn man die Meere gleichmäßig verteilt, sie eine Tiefe von vielleicht
\SI{1000}{m} haben. Das Volumen ist dann:
\[
	V = 4 \piup R_\text{E}^2 \cdot \SI{1000}{m}
	= \SI{5.15e17}{m^3}
\]

Geteilt durch die $\SI{18}{ml/mol} = \SI{1.8e-5}{m^3/mol}$ erhalte ich eine
Stoffmenge von \SI{2.86e22}{mol}, was \num{1.72e46} Teilchen entspricht.

Im Meer ist jetzt ein Anteil von \num{3.89e-22} Markiert. Wenn ich jetzt wieder
\num{6.69e24} Moleküle auswähle, dann ist der Erwartungswert \num{2600}
markierte Moleküle im Glas.

\section{Flugzeit-Massenspektrometer}
\label 2

\subsection{Skizze}

\begin{center}
	\begin{tikzpicture}
		\draw[dashed] (0, 0) -- ++(0, 3);
		\draw[dashed] (2, 0) -- ++(0, 3);
		\node[below] at (1, 0) {Kondensatorplatten};
		\draw[|-|] (0, 3.2) -- ++(2, 0) node[above, midway] {\SI{10}{mm}};
		\node[draw, rectangle, right] at (10, 1.5) {Detektor};
		\draw[|-|] (2, 3.2) -- (10, 3.2) node[above, midway] {\SI{1}{m}};
		\node[draw, circle] at (1, 1.5) {$\mathrm{NO}$};
	\end{tikzpicture}
\end{center}

\subsection{Flugzeit für $\mathrm{^{14}N^{16}O}$}

Die Masse des Isotops ist $\SI{30}{amu} = \SI{4.98e-23}{kg}$. Das Ion legt eine
Potentialdifferenz von \SI{500}{V} zurück (mittig im Kondensator), es bekommt
also $\SI{500}{eV} = \SI{8.05e-17}{J}$ Energie. Mit $E_\text{kin} = mv^2/2$
erhalte ich eine Geschwindigkeit von $v = \SI{56800}{m/s}$. Da die Ruhemasse im
Bereich von \si{MeV} liegt, darf ich an dieser Stelle klassisch rechnen. Mit
dieser Geschwindigkeit $v$ legen die Ionen die Strecke von $L = \SI1m$ in
$t = \SI{1.76e-5}s$ zurück.

Dabei habe ich die Beschleunigungszeit nicht berücksichtigt. Das Ion
beschleunigt innerhalb von $d/2 = \SI5{mm}$ auf die Geschwindigkeit $v$. Mit
$v^2 = 2ad/2$ erhalte ich eine Beschleunigung von $a = \SI{3.23e11}{m/s^2}$.
Die Zeit, um diese Strecke zurück zu legen erhalte ich mit $d/2 = a \tilde
t^2/2$: $\tilde t = \SI{1.76e-7}s$. Der relative Fehler ist also \num{0.01}.

\subsection{Flugzeitverbreiterung}

Für die Flugzeitverbreiterung betrachte ich zwei Ionen, die an beiden Enden des
Wechselwirkungsgebietes gestartet sind. Die Ionen bekommen nun als Energie:
\[
	E_\pm = \frac{\frac d2 \pm \frac{\Deltaup x}2}d e U_0
\]

Diese Energien sind:
\[
	E_+ = \SI{8.13e-17}J
	\eqnsep
	E_- = \SI{7.96e-17}J
\]

Mit den eben benutzten Formeln errechne ich die Flugzeit für beide Ionen aus:
\[
	t_+ = \SI{1.75056e-5}s
	\eqnsep
	t_- = \SI{1.76815e-5}s
\]

Die Zeitauflösung ist also maximal \SI{1.759e-7}s, wenn die endliche Größe des
Wechselwirkungsgebietes berücksichtigt wird. Nun berechne ich die Flugzeit für
die verschiedenen Isotope, wenn sie aus der Mitte des Kondensators starten:
\begin{center}
	\begin{tabular}{lSSS}
		Isotop & {Masse / \si{amu}} & {Flugzeit $t$ / \si s} \\
		\hline
		$\mathrm{^{14}N^{16}O}$ & 30 & 1.75929e-5 \\
		$\mathrm{^{15}N^{16}O}$ & 31 & 1.78837e-5 \\
		$\mathrm{^{14}N^{18}O}$ & 32 & 1.81689e-5 \\
	\end{tabular}
\end{center}

Die Werte unterschieden sich mehr als \SI{2e-7}s, so dass eine Unterscheidung
möglich ist, wenn auch knapp.

\section{Mittlere freie Weglänge}
\label 3

\subsection{Zimmerumgebung}

Gesucht ist die mittlere freie Weglänge $l$. Dazu stelle ich die Gasgleichung um:
\[
	pV = N k_\text{B}  T
	\iff
	p = n k_\text{B}  T
	\iff
	n = \frac p{k_\text{B} T}
\]

Die mittlere freie Weglänge $l$ ist:
\[
	l = \frac 1{n\sigma}
	\iff
	l = \frac{k_\text{B} T}{p\sigma}
	\implies
	l = \SI{1.00e-7}m
\]

Die Teilchendichte ist $n = \SI{2.50e25}{m^{-3}}$. Der mittlere Abstand $a$ zwischen den Teilchen ist dann:
\[
	a = \sqrt[3]{\frac 1n} = \SI{3.42e-9}m
\]

Dieser Abstand ist um zwei Größenordnungen kleiner als die mittlere freie
Weglänge. Dies liegt daran, dass der Wirkungsquerschnitt bei diesem Abstand
einen kleinen Raumwinkel einnimmt.

\subsection{Evakuieren}

Die gesuchte Teilchendichte ist $n = 1/l\sigma$. Nach der Gasgleichung ist dies auch gleich $p/k_\text{B} T$. Nach $p$ aufgelöst:
\[
	p = \frac{k_\text{B} T}{l \sigma} = \SI{10}{mPa}
\]

\section{De Broglie-Wellenlängen}
\label 4

\subsection{Energie \SI{10}{eV}}

Die Energie-Impuls-Relation besagt mit Gesamtenergie $E$, Ruheenergie $E_0$ und
Impuls $p$: $E^2 = E_0^2 + (cp)^2$. Die de Broglie-Wellenlänge eines Teilchens
ist $\lambda = h/p$. Damit errechne ich:
\[
	\lambda = \frac{ch}{\sqrt{\del{E_0 + \Deltaup E}^2 - E_0^2}}
\]

Dort setze ich $\Deltaup E = \SI{10}{eV}$ und $E_0 = \SI{511}{keV}$ ein und
erhalte $\lambda = \SI{3.86e-10}m$.

\subsection{Energie \SI{20}{keV}}

Gleiche Rechnung, nur mit anderer $\Deltaup E = \SI{20}{keV}$. Das Ergebnis ist
$\lambda = \SI{8.56}m$.

\subsection{$\alpha$-Teilchen}

Die Ruhemasse eines $\alpha$-Teilchens ist $m_\alpha = c^{-2} \cdot
\SI{3727}{MeV}$. Ich bestimme die Geschwindigkeit mit der relativistischen
Formel:
\[
	v = \SI{1.70e7}{m/s}
\]

\subsection{Stickstoffmolekül bei Zimmertemperatur}

Die Masse von $\mathrm{^{14}N_2}$ ist \SI{28}{amu}. Bei Zimmertemperatur $T = \SI{290}{K}$ ist die kinetische Energie pro Freiheitsgrad $E = kT/2$. Die mittlere Geschwindigkeit ist somit:
\[
	v
	= \sqrt{2 \frac Em}
	= \sqrt{\frac{kT}m}
	= \SI{293}{m/s}
\]

Der Impuls ist $p = \SI{1.37e-23}{kg.m/s}$, die de Broglie-Wellenlänge dazu ist
$\lambda = \SI{4.84e-11}{m}$.

Der Ablenkwinkel $\alpha$ für das erste Maximum ist mit der Formel aus der Optik:
\[
	\sin\del\alpha = \frac{\lambda}{a}
	\implies
	\alpha = \SI{484}{\micro rad}
\]

\subsection{Schnecke}

Das Gewicht einer Schnecke ist vielleicht $m = \SI{30}{g}$. Die
Kriechgeschwindigkeit ist vielleicht $v = \SI{1}{mm/s}$. Dann ist der Impuls $p
= mv = \SI{3.0e-5}{kg.m/s}$. Der Streuwinkel am Gitter mit $a = \SI{10}{cm}$
ist $\alpha = \SI{2.21e-28}{rad}$, also nicht zu beobachten.

Damit es aber überhaupt wirklich zur Interferenz kommen kann, darf der Ort der
Schnecke nicht mehr gemessen werden, bis sie durch den Gartenzaun ist. Dies ist
allerdings schwer möglich, da sie den Boden (der sie misst) zur Fortbewegung
braucht. Ohne Luft und Licht hat es die Schnecke noch schwerer.

\subsection{Photon}

Die Energie des Photons ist $E = ch/\lambda$. Sein Impuls ist $p = h/\lambda $.
Die de Broglie-Wellenlänge $\lambda_\text{dB}$ ist dann $\lambda$.

\section{Kastenpotential}
\label 5

Ich beginne mit der gegebenen Schrödingergleichung.
\begin{align*}
	\ii \hbar \partial_t \ket{\psi(x, t)} &= \hamilton \ket{\psi(x, t)} \\
	\ii \hbar \partial_t \ket{\psi(x, t)} &= \del{\frac{p^2}{2m} + U(x)} \ket{\psi(x, t)} \\
	\intertext{%
		Der Impulsoperator $\hat p$ ist nach dem, was ich in \cite[Seite
		496]{penrose-road_to_reality} gelesen habe $\hat p = \ii \hbar
		\partial_x$ und nicht proportional zu $\partial_t^2$. So kann ich die
		Schrödingergleichung schreiben als:
	}
	\ii \hbar \partial_t \ket{\psi(x, t)} &= \del{- \frac{\hbar^2}{2m} \partial_x^2 + U(x)} \ket{\psi(x, t)} \\
	\intertext{%
		Das Potential $U(x)$ ist innerhalb des kompakten Intervals identisch
		null, so dass ich diesen Summanden weglassen kann, wenn ich das Problem
		nur auf diesem Interval betrachte.
	}
	\ii \hbar \partial_t \ket{\psi(x, t)} &= -\frac{\hbar^2 }{2m} \partial_x^2 \ket{\psi(x, t)} &x \in \intcc{-\frac a2, \frac a2} \\
	\ii \partial_t \ket{\psi(x, t)} &= -\frac\hbar{2m} \partial_x^2 \ket{\psi(x, t)} &x \in \intcc{-\frac a2, \frac a2}
\end{align*}

Diese parabolische partielle Differentialgleichung auf einem kompakten Interval
löse ich mit einem Separationsansatz. Mein Ansatz ist: $\psi(x, t) = \phi(x)
\theta(t)$. Damit wird die Gleichung zu:
\begin{align*}
	\ii \phi(x) \dot \theta(t) &= -\frac{\hbar^2}{2m} \phi''(x) \theta(t) &x \in \intcc{-\frac a2, \frac a2} \\
	\ii \frac{\dot \theta}\theta &= -\frac{\hbar^2}{2m} \frac{\phi''}\phi = \alpha^2 &x \in \intcc{-\frac a2, \frac a2}
\end{align*}

Die Integralbasis für $\phi$ besteht aus folgenden Elementen:
\[
	\set{
		\cos\del{\frac\hbar{\sqrt{2m}}\alpha x},
		\sin\del{\frac\hbar{\sqrt{2m}}\alpha x}
	}
\]

Die Integralbasis für $\theta$ dagegen ist:
\[
	\set{
		\exp\del{
			- \ii \alpha^2
		}
	}
\]

Das Teilchen darf sich außerhalb des Intervalls nicht aufhalten, da es dafür
dann eine unendliche Energie bräuchte. Daher muss die Wellenfunktion dort null
sein. Wegen der geforderten Stetigkeit von $\psi$ muss an den Randpunkten
$\psi\del{\pm a/2} = 0$ gelten.

Mit dieser Randbedingung kann ich nun $\alpha$ näher bestimmen. Es muss für den
Kosinus gelten:
\[
	\frac\hbar{\sqrt{2m}}\alpha \frac a2 = \del{n + \half} \piup
	\iff
	\alpha = \frac{(2n+1)\piup \sqrt{2m}}{a\hbar}
\]

Für den Sinus:
\[
	\frac\hbar{\sqrt{2m}}\alpha \frac a2 = n\piup
	\iff
	\alpha = \frac{2n\piup \sqrt{2m}}{a\hbar}
\]

Somit wird die Integralbasis für $\phi$ zu:
\[
	\set{
		\cos\del{\frac{(2n+1)\piup}{a} x},
		\sin\del{\frac{2n\piup}{a} x},
	}
\]

\subsection{Mögliche Impulse}

Die Wellenzahlen $k$, die die Welle annehmen darf sind dann $n\piup/a$, je nach dem, ob es eine Sinus- oder Kosinuswelle ist.

\subsection{Energie der Welle}

Die Energie $E$ ist ein Eigenwert des Energieoperators $\iup \hbar \partial_t$. Ich nehme mir den entsprechenden Teil aus der Schrödingergleichung:
\begin{align*}
	E \ket\psi &= \iup \hbar \partial_t \ket\psi \\
	\del{E - \ii \hbar \partial_t} \ket\psi &= 0 \\
	\del{E - \ii \hbar \partial_t} \cos(kx) \exp\del{\ii \frac{(2n+1)\piup}{\sqrt{2m}\hbar} t} &= 0
\end{align*}

Beim Ableiten nach der Zeit $t$ erhalte ich die innere Ableitung der Exponentialfunktion. Multipliziert mit dem Vorfaktor erhalte ich:
\[
	E = - \frac{(2n+1)\piup}{\sqrt{2m}a}
\]

Der Betrag der Energie $\abs E$ wird mit steigender Bauchzahl $n$ größer, allerdings sollte die Energie größer und nicht kleiner werden.

\textcolor{darkred}{%
	In dieser Teilaufgabe habe ich noch den älteren Wert für $\alpha$, so dass
	diese Aufgabe eigentlich komplett neu gerechnet werden müsste.
}

\subsection{Normierung}

\fehlt

%%%%%%%%%%%%%%%%%%%%%%%%%%%%%%%%%%%%%%%%%%%%%%%%%%%%%%%%%%%%%%%%%%%%%%%%%%%%%%%
%                                    Ende                                     %
%%%%%%%%%%%%%%%%%%%%%%%%%%%%%%%%%%%%%%%%%%%%%%%%%%%%%%%%%%%%%%%%%%%%%%%%%%%%%%%

\IfFileExists{\bibliographyfile}{
	\bibliography{\bibliographyfile}
}{}

\end{document}

% vim: spell spelllang=de
