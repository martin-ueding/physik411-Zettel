% Copyright © 2013 Martin Ueding <dev@martin-ueding.de>
%
\input{header.tex}

\usepackage{pdfpages}
\usepackage{tikz}
\usepackage{cleveref}
\usetikzlibrary{calc}

\newcommand{\themodul}{physik411}
\newcommand{\thegruppe}{Gruppe 2 -- Florian Seidler}
\newcommand{\theuebung}{10}

\ifoot{\footnotesize{Martin Ueding}}
\ihead{\themodul{} -- Übung \theuebung}
\ofoot{\footnotesize{\thegruppe}}

\def\thesubsection{\thesection\alph{subsection}}

\title{\themodul{} -- Übung \theuebung}
\subtitle{\thegruppe}
\author{
	Martin Ueding \footnote{\href{mailto:mu@uni-bonn.de}{mu@uni-bonn.de}}
}

\hypersetup{
	pdftitle={\themodul {} - Übung \theuebung},
}

\begin{document}

\maketitle

\begin{center}
	\ccbysadetitle
\end{center}

%%%%%%%%%%%%%%%%%%%%%%%%%%%%%%%%%%%%%%%%%%%%%%%%%%%%%%%%%%%%%%%%%%%%%%%%%%%%%%%
%                      Dirac-Singularitäten in Graphen                       %
%%%%%%%%%%%%%%%%%%%%%%%%%%%%%%%%%%%%%%%%%%%%%%%%%%%%%%%%%%%%%%%%%%%%%%%%%%%%%%%

\section{Dirac-Singularitäten in Graphen}
\label 1

\subsection{Hybridorbital}

Die $\sigmaup$-Bindung liegt in der Ebene, die $\piup$-Bindung liegt senkrecht
dazu. Die Bindungen mit $^*$ sind die Antibindenen.

\subsection{Hybridisierungstyp}

Da noch ein $\piup$-Bindung übrig bleibt, und zwei p-Orbitale eine mit dem
s-Orbital entartete Bindungsenergie haben, muss es $\mathrm{sp^2}$ sein. Dieses
Hybridorbital ist in \cref{fig:sp2} dargestellt.

\begin{figure}
	\centering
	\includegraphics[width=.5\textwidth]{sp2.png}
	\caption{%
		$\mathrm{sp^2}$-Hybridorbital. Bild aus \cite{Winter/Orbitron/sp2}.
	}
	\label{fig:sp2}
\end{figure}

\subsection{Besetzung}

Kohlenstoff hat 4 Valenzelektronen. So kann jeweils ein Elektron in eines der
Hybridorbitale und eins in das $p_z$-Orbital. Jedes der Hybridorbitale geht
eine $\sigmaup$-Bindung mit einem benachbarten Atom ein, so kommt jeweils noch
ein weiteres Elektron mit entgegengesetztem Spin dazu. Das Elektron im
$p_z$-Orbital besetzt das $\piup$-Molekülorbital. Es geht mit allen Nachbarn
gleichzeitig eine $\piup$-Bindung ein, jedoch nur eine Gleichzeitig. Dies führt
zu einem delokalisierten $\piup$-Molekülorbital.

Die Besetzung ist dann also:
\begin{gather*}
	\underset{\sigmaup^*}{\boxed{\phantom{\uparrow\downarrow}}\boxed{\phantom{\uparrow\downarrow}}\boxed{\phantom{\uparrow\downarrow}}} \\
	\underset{\piup^*}{\boxed{\phantom{\uparrow\downarrow}}} \\
	\underset{\piup}{\boxed{\uparrow\downarrow}} \\
	\underset{\sigmaup}{\boxed{\uparrow\downarrow}\boxed{\uparrow\downarrow}\boxed{\uparrow\downarrow}}
\end{gather*}

Also ist $\sigmaup$ mit 6 Elektronen besetzt, und $\piup$ mit 2, der Rest leer.
Daher ist die Bindung auch maximal stark.

\subsection{Bloch-Funktionen}

Die vier Symmetrieachsen sind schon in Abbildung (1a) auf dem Aufgabenzettel
eingezeichnet. So sind die Spiegelungen an Achsen, die senkrecht zur $z$-Achse
sind, die Achsen $C2'$, $C2''$ und $C2''$. Die Spiegelung an der $x$-$y$-Ebene
ist wohl durch $C3$ gegeben, wobei das auch die Punktsymmetrie des Kristalls
sein könnte.

Die drei Spiegelungen $C2$ sind wohl von gerader Parität und um
\SI{120}{\degree} versetzt, so dass diese unterschiedlich sind. Die Spiegelung
$C3$ ist von ungerader Parität, da das $p_z$ Orbital selbst antisymmetrisch
ist.

Die einzelnen Orbitale s und p haben andere Symmetrien, das s-Orbital ist
kugelsymmetrisch und kann daher beliebig gespiegelt oder gedreht werden. Die
$p_i$-Orbitale können an der $i$-Achse symmetrisch gespiegelt werden, an den
anderen beiden Achsen jedoch nur antisymmetrisch.

Die $\sigmaup$ und $\piup$ Bahnen sind nicht gekoppelt, da sie getrennte
Orbitale sind. Genauso wie schon s und p orthogonale Wellenfunktionen sind.
Durch die Hybridisierung bleibt die Anzahl der Wellenfunktionen gleich, so dass
diese weiterhin orthogonal sein können.

\subsection{Äquivalenz von Hochsymmetriepunkten}

Die Punkte $K_1$ und $K_5$ sowie $K_2$ und $K_4$ lassen sich durch $\vec g_2$
aufeinander abbilden. $\vec g_1$ tut dies mit $K_4$ und $K_6$. Die Summe der
beiden Gittervektoren lässt noch $K_3$ und $K_5$ äquivalent sein. Letztlich
gilt:
\[
	K_i \sim K_j
	\iff
	|i - j| \mod 2 = 0
\]

Damit gibt sich die eine Gruppe, in der $i$ und $j$ gerade sind, sowie eine
andere Gruppe, in der beide ungerade sind.

\subsection{Punktgruppe D3h}

Die Drehung $C_3$ um \SI{120}{\degree} funktioniert auch im reziproken Gitter,
sie bildet gerade auf gerade Punkte ab. Die Rotationen $C_2$ bildet auch
äquivalente $K$ aufeinander ab, wenn die Rotationsachse genau gleich im Raum
liegt. Dann bildet $C_2'''$ $K_6$ auf $K_4$ ab, sowie $K_1$ auf $K_3$ ab.

\subsection{Überlagerung von Bloch-Funktionen}

Das Bloch-Theorem besagt, dass die Wellenfunktionen aus einem periodischen
Potential, das die gleiche Periodizität wie das Kristallgitter hat, und einer
ebenen Welle besteht. Jede Überlagerung hat schon mal eine derartige ebene
Welle:
\[
	\psi_{\vec k} (\vec r)
	= 
	\sum_{n_1, n_2} \del{c_1 \psi_{p_z} (\vec r - n_1 \vec t_1 - n_2 \vec t_2) + c_2 \psi_{p_z} (\vec r - n_1 \vec t_1 - n_2 \vec t_2 - \vec d)} \exp\del{\ii \vec k \cdot (n_1 \vec t_1 + n_2 \vec t_2)}
\]

Durch die Summation über die Basisvektoren $\vec t$ ist jeder Summand in der
Klammer genauso periodisch wie das Gitter selbst. Auch wenn der zweite Summand
zum ersten Verschoben ist, ist die Periodizität immer noch die gleiche, wie das
Gitter. Daher erfüllt $\psi_{\vec k}(\vec r)$ das Bloch-Theorem.

\subsection{Energieeigenwerte}

Die Funktionen $\psi_{\vec k, A}(\vec r)$ und $\psi_{\vec k, B}(\vec r)$ bilden
die Basis für den Hamiltonoperator $\hat H_k$. Die Eigenwerte dieses Operators
sind die Energien, die im Leitungs- und Valenzband herauskommen. Daher brauche
ich nur die Eigenwerte $\lambda$ dieser Matrix zu bestimmen.
\begin{align*}
	\det \begin{pmatrix}
		E_0 - \lambda & J\cdot\del{1 + \exp(-\ii \vec k \vec t_1) + \exp(-\ii \vec k \vec t_2)} \\
		J\cdot\del{1 + \exp(\ii \vec k \vec t_1) + \exp(\ii \vec k \vec t_2)} & E_0 - \lambda
	\end{pmatrix}
	&= 0 \\
	(E_0 - \lambda)^2 - J^2\cdot \del{1 + \exp(\ii \vec k \vec t_1) + \exp(\ii \vec k \vec t_2)} \del{1 + \exp(-\ii \vec k \vec t_1) + \exp(-\ii \vec k \vec t_2)} &= 0 \\
	\lambda^2 - 2 E_0 \lambda + E_0^2 - J^2\cdot \del{3 + 2\cos(\vec k \vec t_1) + 2 \cos(\vec k \vec t_2) + 2 \cos\del{\vec k (\vec t_1 - \vec t_2)}} &= 0
\end{align*}

Die Lösung für die Eigenwerte, und damit für die Eigenenergien, ist:
\[
	\lambda = E_0 \pm J \sqrt{3 + 2\cos(\vec k \vec t_1) + 2 \cos(\vec k \vec t_2) + 2 \cos\del{\vec k (\vec t_1 - \vec t_2)}}
\]

\subsection{Entartung}

Ich setze:
\[
	\vec k := \frac{\vec g_1 - \vec g_2}3
\]

Mit $\vec g_i \vec t_j = 2 \piup \delta_{ij}$ folgt dann für $\lambda$:
\begin{align*}
	\lambda
	&= E_0 \pm J \sqrt{3 + 2\cos(2\piup/3) + 2 \cos(-2\piup/3) + 2 \cos(4\piup/3)} \\
	&= E_0 \pm J \times 0 \\
	&= E_0
\end{align*}

Somit haben beide ($\pm$) Energien den gleichen Wert und sind entartet.

%%%%%%%%%%%%%%%%%%%%%%%%%%%%%%%%%%%%%%%%%%%%%%%%%%%%%%%%%%%%%%%%%%%%%%%%%%%%%%%
%                Zustandsdichte im zweidimensionalen Systemen                 %
%%%%%%%%%%%%%%%%%%%%%%%%%%%%%%%%%%%%%%%%%%%%%%%%%%%%%%%%%%%%%%%%%%%%%%%%%%%%%%%

\section{Zustandsdichte im zweidimensionalen Systemen}

\subsection{Zustandsdichte}

\begin{align*}
	D(E)
	&= \sum_{\vec k, s} \delta\del{E - E_\text{frei} \del{|k|}} \\
	\intertext{%
		Dies wandele ich mit der auf dem Aufgabenblatt gegebenen Formel in ein
		Integral um. Dabei ist die Dimension $m = 1$.
	}
	&= \frac{S}{2\piup} \int_{-\infty}^\infty \dif k \, \delta\del{E - E_\text{frei} \del{|k|}} \\
	\intertext{%
		Die $\delta$-Distribution hat nun die Eigenschaft bei Verkettung, dass
		eine Summe über alle Nullstellen entsteht. Daher formt sich der
		Ausdruck um zu:
	}
	&= \frac{S}{2\piup} \sum_n \int_{-\infty}^\infty \dif k \, \frac{\delta(k-k_n)}{\od{E_\text{frei}}k} \\
	\intertext{%
		Die Dispersionsrelation ist streng monoton steigend, somit dann auf dem
		relevanten Intervall bijektiv. Daher gibt es nur eine Nullstelle der
		inneren Funktion, $k - k_0$. Die Summe entfällt.
	}
	&= \frac{S}{2\piup} \int_{-\infty}^\infty \dif k \, \frac{\delta(k-k_0)}{\od{E_\text{frei}}k} \\
	\intertext{%
		Und da die Dispersionsrelation nur vom Betrag abhängt, kann die
		Integration auch von 0 begonnen werden. Allerdings sollte dann noch ein
		Faktor 2 dazu.
	}
	&= \frac{S}{\piup} \int_0^\infty \dif k \, \frac{\delta(k-k_0)}{\od{E_\text{frei}}k} \\
\end{align*}

\subsection{Explizite Formel}

Ich setze ein:
\[
	E_\text{frei}(K) = \frac{\hbar^2 k^2}{2m}
\]

Die Nullstelle der Funktion ist bei:
\[
	k_0 = \frac{\sqrt{2mE}}\hbar
\]

Wenn ich diese auch noch in die $D(E)$ einsetze und integriere, erhalte ich:
\[
	D(E) = \frac S\piup \frac{\hbar{\sqrt{2E}}}{\sqrt m}
\]

\subsection{Anzahl der Quantenzustände}

Ich integriere die Dichte nach dem Wellenvektor von $0$ bis $E_\text{F}$ und
erhalte dann die Gesamtzahl an Zuständen. Ich erhalte:
\[
	\frac{2S\hbar}{3\piup} \sqrt{\frac2m} E_\text{F}^{3/2}
\]

Da die Teilchen Spin $1/2$ haben, sind es Fermionen. Ich muss also den Rest der
Teilaufgabe rechnen. Die gerade errechnete Anzahl setze ich gleich $n$ und löse
nach $E_\text{F}$ auf. Ich erhalte:
\[
	E_\text{F} = \del{\frac{3 n \piup}{2 S \hbar} \sqrt{\frac m2}}^{2/3}
\]

Jedoch ist diese Energie gleich 0, wenn die Masse $m$ gleich 0 ist.

\subsection{Masseloses Teilchen}

Die Fermi-Geschwindigkeit ist:
\[
	v_\text{F} = \sqrt{\frac{2E_\text{F}}{m_\text{e}}}
\]

Wenn ich allerdings $m = 0$ in die Formel aus der vorherigen Teilaufgabe
einsetze, erhalte ich einfach nur 0.

%%%%%%%%%%%%%%%%%%%%%%%%%%%%%%%%%%%%%%%%%%%%%%%%%%%%%%%%%%%%%%%%%%%%%%%%%%%%%%%
%                                    Ende                                     %
%%%%%%%%%%%%%%%%%%%%%%%%%%%%%%%%%%%%%%%%%%%%%%%%%%%%%%%%%%%%%%%%%%%%%%%%%%%%%%%

\IfFileExists{\bibliographyfile}{
	\bibliography{\bibliographyfile}
}{}

\end{document}

% vim: spell spelllang=de
