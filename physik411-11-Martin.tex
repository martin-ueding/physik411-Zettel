% Copyright © 2013 Martin Ueding <dev@martin-ueding.de>
%
\input{header.tex}

\usepackage{pdfpages}
\usepackage{tikz}
\usepackage{cleveref}
\usetikzlibrary{calc}

\newcommand{\themodul}{physik411}
\newcommand{\thegruppe}{Gruppe 2 -- Florian Seidler}
\newcommand{\theuebung}{10}

\ifoot{\footnotesize{Martin Ueding}}
\ihead{\themodul{} -- Übung \theuebung}
\ofoot{\footnotesize{\thegruppe}}

\def\thesubsection{\thesection\alph{subsection}}

\title{\themodul{} -- Übung \theuebung}
\subtitle{\thegruppe}
\author{
	Martin Ueding \footnote{\href{mailto:mu@uni-bonn.de}{mu@uni-bonn.de}}
}

\hypersetup{
	pdftitle={\themodul {} - Übung \theuebung},
}

\begin{document}

\maketitle

\begin{center}
	\ccbysadetitle
\end{center}

\begin{table}[h]
	\centering
	\begin{tabular}{l*3{|c}}
		Aufgabe
		& \ref 1
		& \ref 2
		& $\sum$   \\
		\hline
		Punkte
		& \punkte / 14
		& \punkte / 10
		& \punkte / 40
	\end{tabular}
\end{table}

%%%%%%%%%%%%%%%%%%%%%%%%%%%%%%%%%%%%%%%%%%%%%%%%%%%%%%%%%%%%%%%%%%%%%%%%%%%%%%%
%                      Dirac-Singularitäten in Graphen                       %
%%%%%%%%%%%%%%%%%%%%%%%%%%%%%%%%%%%%%%%%%%%%%%%%%%%%%%%%%%%%%%%%%%%%%%%%%%%%%%%

\section{Dirac-Singularitäten in Graphen}
\label 1

\subsection{Hybridorbital}

\begin{figure}
	\centering
	\includegraphics[width=.5\textwidth]{sp2.png}
	\caption{}
	\label{fig:sp2}
\end{figure}

%%%%%%%%%%%%%%%%%%%%%%%%%%%%%%%%%%%%%%%%%%%%%%%%%%%%%%%%%%%%%%%%%%%%%%%%%%%%%%%
%                                    Ende                                     %
%%%%%%%%%%%%%%%%%%%%%%%%%%%%%%%%%%%%%%%%%%%%%%%%%%%%%%%%%%%%%%%%%%%%%%%%%%%%%%%

\IfFileExists{\bibliographyfile}{
	\bibliography{\bibliographyfile}
}{}

\end{document}

% vim: spell spelllang=de
