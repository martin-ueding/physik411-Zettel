% Copyright © 2013 Martin Ueding <dev@martin-ueding.de>
%
\input{header.tex}

\usepackage{tikz}

\newcommand{\themodul}{physik411}
\newcommand{\thegruppe}{Gruppe 2 -- Florian Seidler}
\newcommand{\theuebung}{4}

\ifoot{\footnotesize{Martin Ueding}}
\ihead{\themodul{} -- Übung \theuebung}
\ofoot{\footnotesize{\thegruppe}}

\def\thesubsection{\thesection\alph{subsection}}

\title{\themodul{} -- Übung \theuebung}
\subtitle{\thegruppe}
\author{
	Martin Ueding \footnote{\href{mailto:mu@uni-bonn.de}{mu@uni-bonn.de}}
}

\hypersetup{
	pdftitle={\themodul {} - Übung \theuebung},
}

\begin{document}

\maketitle

\begin{center}
	\ccbysadetitle
\end{center}

\begin{Form}
\begin{table}[h]
	\centering
	\begin{tabular}{l|c|c|c|c}
		Aufgabe
		& \ref 1
		& \ref 2
		& \ref 3
		& $\sum$   \\
		\hline
		Punkte
		& \TextField[name=aufgabe1, width=1cm]{} / 3
		& \TextField[name=aufgabe2, width=1cm]{} / 5
		& \TextField[name=aufgabe3, width=1cm]{} / 15
		& \TextField[name=ergebnis, width=1cm]{} / 23
	\end{tabular}
\end{table}
\end{Form}

%%%%%%%%%%%%%%%%%%%%%%%%%%%%%%%%%%%%%%%%%%%%%%%%%%%%%%%%%%%%%%%%%%%%%%%%%%%%%%%
%                         Separation von Koordinaten                          %
%%%%%%%%%%%%%%%%%%%%%%%%%%%%%%%%%%%%%%%%%%%%%%%%%%%%%%%%%%%%%%%%%%%%%%%%%%%%%%%

\section{Separation von Koordinaten}
\label 1

Zeitunabhängige Schrödingergleichung:
\[
	E \ket\psi = \hat H \ket \psi
	\eqnsep
	\hat H = \frac{\hat p^2}{2m} + \frac\kappa r
	\eqnsep
	\hat p^2 = - \hbar^2 \laplace
\]

In Kugelkoordinaten:
\[
	\laplace = \frac1{r^2} \dpd{}r r^2 \dpd{}r + \frac{1}{r^2 \sin\del\theta} \dpd{}\theta \sin\del\theta \dpd{}\theta + \frac1{r^2\sin^2\del\theta} \dpd[2]{}\phi
\]

Mit dem Separationsansatz $\psi(\vec x) = R(r) \Theta(\theta) \Phi(\phi)$ kann
ich die Variable $r$ separieren:
\[
	r^2 E - \kappa r - \dpd{}r r^2 R'(r) = \frac1{\sin\del\theta} \dpd{}\theta \sin\del\theta \Theta'(\theta) + \frac1{\sin^2\del\theta} \Phi''(\phi)
\]

Und als letzten Schritt noch $\theta$ und $\phi$:
\[
	\sin\del\theta \dpd{}\theta \sin\del\theta \Theta'(\theta) = - \Phi''(\phi)
\]

%%%%%%%%%%%%%%%%%%%%%%%%%%%%%%%%%%%%%%%%%%%%%%%%%%%%%%%%%%%%%%%%%%%%%%%%%%%%%%%
%                           Kugelflächenfunktionen                           %
%%%%%%%%%%%%%%%%%%%%%%%%%%%%%%%%%%%%%%%%%%%%%%%%%%%%%%%%%%%%%%%%%%%%%%%%%%%%%%%

\section{Kugelflächenfunktionen}
\label 2

\fehlt

%%%%%%%%%%%%%%%%%%%%%%%%%%%%%%%%%%%%%%%%%%%%%%%%%%%%%%%%%%%%%%%%%%%%%%%%%%%%%%%
%                           Zirkulare Rydberg-Atome                           %
%%%%%%%%%%%%%%%%%%%%%%%%%%%%%%%%%%%%%%%%%%%%%%%%%%%%%%%%%%%%%%%%%%%%%%%%%%%%%%%

\section{Zirkulare Rydberg-Atome}
\label 3

\fehlt

%%%%%%%%%%%%%%%%%%%%%%%%%%%%%%%%%%%%%%%%%%%%%%%%%%%%%%%%%%%%%%%%%%%%%%%%%%%%%%%
%                                    Ende                                     %
%%%%%%%%%%%%%%%%%%%%%%%%%%%%%%%%%%%%%%%%%%%%%%%%%%%%%%%%%%%%%%%%%%%%%%%%%%%%%%%

\IfFileExists{\bibliographyfile}{
	%\bibliography{\bibliographyfile}
}{}

\end{document}

% vim: spell spelllang=de
