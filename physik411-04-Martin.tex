% Copyright © 2013 Martin Ueding <dev@martin-ueding.de>
%
\input{header.tex}

\usepackage{tikz}

\newcommand{\themodul}{physik411}
\newcommand{\thegruppe}{Gruppe 2 -- Florian Seidler}
\newcommand{\theuebung}{4}

\ifoot{\footnotesize{Martin Ueding}}
\ihead{\themodul{} -- Übung \theuebung}
\ofoot{\footnotesize{\thegruppe}}

\def\thesubsection{\thesection\alph{subsection}}

\title{\themodul{} -- Übung \theuebung}
\subtitle{\thegruppe}
\author{
	Martin Ueding \footnote{\href{mailto:mu@uni-bonn.de}{mu@uni-bonn.de}}
}

\hypersetup{
	pdftitle={\themodul {} - Übung \theuebung},
}

\begin{document}

\maketitle

\begin{center}
	\ccbysadetitle
\end{center}

\begin{Form}
\begin{table}[h]
	\centering
	\begin{tabular}{l|c|c|c|c}
		Aufgabe
		& \ref 1
		& \ref 2
		& \ref 3
		& $\sum$   \\
		\hline
		Punkte
		& \TextField[name=aufgabe1, width=1cm]{} / 3
		& \TextField[name=aufgabe2, width=1cm]{} / 5
		& \TextField[name=aufgabe3, width=1cm]{} / 15
		& \TextField[name=ergebnis, width=1cm]{} / 23
	\end{tabular}
\end{table}
\end{Form}

%%%%%%%%%%%%%%%%%%%%%%%%%%%%%%%%%%%%%%%%%%%%%%%%%%%%%%%%%%%%%%%%%%%%%%%%%%%%%%%
%                         Separation von Koordinaten                          %
%%%%%%%%%%%%%%%%%%%%%%%%%%%%%%%%%%%%%%%%%%%%%%%%%%%%%%%%%%%%%%%%%%%%%%%%%%%%%%%

\section{Separation von Koordinaten}
\label 1

Zeitunabhängige Schrödingergleichung:
\[
	E \ket\psi = \hat H \ket \psi
	\eqnsep
	\hat H = \frac{\hat p^2}{2m} + \frac\kappa r
	\eqnsep
	\hat p^2 = - \hbar^2 \laplace
\]

In Kugelkoordinaten:
\[
	\laplace = \frac1{r^2} \dpd{}r r^2 \dpd{}r + \frac{1}{r^2 \sin\del\theta} \dpd{}\theta \sin\del\theta \dpd{}\theta + \frac1{r^2\sin^2\del\theta} \dpd[2]{}\phi
\]

Mit dem Separationsansatz $\psi(\vec x) = R(r) \Theta(\theta) \Phi(\phi)$ kann
ich die Variable $r$ separieren:
\[
	r^2 E - \kappa r - \dpd{}r r^2 R'(r) = \frac1{\sin\del\theta} \dpd{}\theta \sin\del\theta \Theta'(\theta) + \frac1{\sin^2\del\theta} \Phi''(\phi)
\]

Und als letzten Schritt noch $\theta$ und $\phi$:
\[
	\sin\del\theta \dpd{}\theta \sin\del\theta \Theta'(\theta) = - \Phi''(\phi)
\]

%%%%%%%%%%%%%%%%%%%%%%%%%%%%%%%%%%%%%%%%%%%%%%%%%%%%%%%%%%%%%%%%%%%%%%%%%%%%%%%
%                           Kugelflächenfunktionen                           %
%%%%%%%%%%%%%%%%%%%%%%%%%%%%%%%%%%%%%%%%%%%%%%%%%%%%%%%%%%%%%%%%%%%%%%%%%%%%%%%

\section{Kugelflächenfunktionen}
\label 2

\fehlt

%%%%%%%%%%%%%%%%%%%%%%%%%%%%%%%%%%%%%%%%%%%%%%%%%%%%%%%%%%%%%%%%%%%%%%%%%%%%%%%
%                           Zirkulare Rydberg-Atome                           %
%%%%%%%%%%%%%%%%%%%%%%%%%%%%%%%%%%%%%%%%%%%%%%%%%%%%%%%%%%%%%%%%%%%%%%%%%%%%%%%

\section{Zirkulare Rydberg-Atome}
\label 3

\subsection{Effektives Potential und Kreisbahn}

In der Vorlesung wurde folgende Formel gegeben:
\[
	V_\text{eff}(r, l) = \frac{\hbar^2 l(l+1)}{2mr^2} + V_\text{Coul}(r)
\]

Dies setze ich ein:
\[
	V_\text{eff}(r, l) = \frac{\hbar^2 l(l+1)}{2m} \frac1{r^2} - \frac{e^2}{4\piup\varepsilon_0} \frac 1r
\]

Mit $E_\text{Ryd} = e^2 / (8 \piup \varepsilon_0 a_0)$ vereinfacht sich dies
zu:
\[
	V_\text{eff}(r, l) = E_\text{Ryd} \del{l(l+1) \frac{a_0^2}{r^2} - 2 \frac{a_0}r}
\]

Für einige $l$ ist dies in Abbildung \ref{fig:1} dargestellt. Für $r \to 0$
divergiert das Potential gegen $\infty$, für $r \to \infty$ konvergiert das
Potential von unten gegen 0.

\begin{figure}
	\centering
	\includegraphics[width=\textwidth]{3-Potential.pdf}
	\caption{$V_\text{eff}(r, l)$ für $n = 1, \ldots, 10$ und $l = n - 1$}
	\label{fig:1}
\end{figure}

Eine Kreisbahn ist dann erreicht, wenn das Gesamtenergie (im Diagramm eine
horizontale Linie) sich nur einmal mit dem Potential schneidet, also das
Teilchen nur die minimale Energie hat. Dazu bestimme ich das Minimum des
Potentials:
\[
	r = l(l+1) a_0
\]

Dies löse ich nach $l$ auf und setze es in $V_\text{eff}(r, l)$ ein, um die
Ortskurve der Minima zu erhalten. Diese Ortskurve ist in Abbildung \ref{fig:2}
geplottet.

\begin{figure}
	\centering
	\includegraphics[width=\textwidth]{3-Ortskurve.pdf}
	\caption{%
		In blau: $V_\text{eff}(r)$ für $n = 1, \ldots, 5$ und $l = n - 1$. In
		violett: Ortskurve der Minima.
	}
	\label{fig:2}
\end{figure}

\subsection{Große $n$}

Mit $l = n - 1$ ergibt sich $l(l+1) \approx n^2$. Der Radius ist also $n^2 a_0$, wie schon oben hergeleitet.

Die Energie ist dann:
\[
	V_\text{eff}\del{r_\text{zirk}(n)}
	\approx - E_\text{Ryd} \del{\half \frac1{n^2} - \frac1{n^2}}
	= - \half E_\text{Ryd} \frac1{n^2}
\]

\subsection{Überlagerung}

Für die Überlagerung betrachte ich zwei Zustände mit einer relativen Amplitude
von $1/\sqrt2$. Die Energie im Zustand $n$, $E_n$, war in der Vorlesung
angegeben mit $-E_\text{Ryd}/n^2$. Die Zeitabhängigkeit ist gerade
$\exp\del{-\ii Et/\hbar}$, durch die Separation der Zeit mit Eigenwert $E$.
\[
	\frac1{\sqrt2} \exp\del{\ii (n-1) \phi} \exp\del{\ii \frac{E_\text{Ryd}}{\hbar n^2} t}
	+
	\frac1{\sqrt2} \exp\del{\ii (n-2) \phi} \exp\del{\ii \frac{E_\text{Ryd}}{\hbar (n-1)^2} t}
\]

Die Wahrscheinlichkeitsdichte dazu ist:
\[
	\varrho(\phi, t)
	= 1 + \cos\del{\phi + \frac{E}\hbar \del{\frac1{n^2} - \frac1{(n-1)^2}}t}
\]

Diese hat die Wellenzahl $k = 1$ und hat somit genau eine Mode. Da dieser
Zustand dadurch entstanden sind, dass zwei Wellen, die sich durch eine Periode
mehr unterschieden haben, überlagert worden ist, entsteht eine Schwebung mit
der Differenz der Frequenzen. Diese Differenz ist allerdings gerade eine
Periode.

Die Kreisfrequenz ist:
\begin{align*}
	\omega
	&= - \frac{E_\text{Ryd}}\hbar \del{\frac1{n^2} - \frac1{(n-1)^2}} \\
	&= - \frac{E_\text{Ryd}}\hbar \del{\frac1{n^2} - \frac1{n^2 - 2n + 1}} \\
	&= - \frac{E_\text{Ryd}}\hbar \frac{-2n + 1}{n^2 \del{n^2 - 2n + 1}} \\
	&\approx \frac{2E_\text{Ryd}}\hbar {n^3} \\
\end{align*}

Klassisch wende ich wieder „Coulombanziehung wirkt als Zentripetalkraft“ an:
\[
	\frac{1}{4\piup \varepsilon_0} \frac{r^2}{r^2} = m \omega^2 r
\]

Mit $r = a_0 n^2$:
\begin{align*}
	\omega &= \sqrt{\frac{1}{4\piup \varepsilon_0} \frac{e^2}{a_0^3 n^6 m}} \\
					   &= \sqrt{\frac{E_\text{Ryd}}{2a_0^2m}} \frac{1}{n^3}
\end{align*}

Es kommt also die gleiche Abhängigkeit von $n$ heraus, allerdings ist der
Vorfaktor ein anderer. Von den Einheiten kommt allerdings in beiden Fällen
\si{\radian\per\second} heraus.

\fehlt

\subsection{Empfindlichkeit}

\fehlt

%%%%%%%%%%%%%%%%%%%%%%%%%%%%%%%%%%%%%%%%%%%%%%%%%%%%%%%%%%%%%%%%%%%%%%%%%%%%%%%
%                                    Ende                                     %
%%%%%%%%%%%%%%%%%%%%%%%%%%%%%%%%%%%%%%%%%%%%%%%%%%%%%%%%%%%%%%%%%%%%%%%%%%%%%%%

\IfFileExists{\bibliographyfile}{
	%\bibliography{\bibliographyfile}
}{}

\end{document}

% vim: spell spelllang=de
