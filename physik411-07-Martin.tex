% Copyright © 2013 Martin Ueding <dev@martin-ueding.de>
%
\input{header.tex}

\usepackage{tikz}
\usetikzlibrary{calc}
\usepackage{xfrac}

\newcommand{\themodul}{physik411}
\newcommand{\thegruppe}{Gruppe 2 -- Florian Seidler}
\newcommand{\theuebung}{7}

\ifoot{\footnotesize{Martin Ueding}}
\ihead{\themodul{} -- Übung \theuebung}
\ofoot{\footnotesize{\thegruppe}}

\def\thesubsection{\thesection\alph{subsection}}

\title{\themodul{} -- Übung \theuebung}
\subtitle{\thegruppe}
\author{
	Martin Ueding \footnote{\href{mailto:mu@uni-bonn.de}{mu@uni-bonn.de}}
}

\hypersetup{
	pdftitle={\themodul {} - Übung \theuebung},
}

\begin{document}

\maketitle

\begin{center}
	\ccbysadetitle
\end{center}

\begin{table}[h]
	\centering
	\begin{tabular}{l|c|c|c|c|c}
		Aufgabe
		& \ref 1
		& \ref 2
		& \ref 3
		& \ref 4
		& $\sum$   \\
		\hline
		Punkte
		& \punkte / 8
		& \punkte / 12
		& \punkte / 14
		& \punkte / 0
		& \punkte / 34
	\end{tabular}
\end{table}

%%%%%%%%%%%%%%%%%%%%%%%%%%%%%%%%%%%%%%%%%%%%%%%%%%%%%%%%%%%%%%%%%%%%%%%%%%%%%%%
%                              Hund'sche Regeln                               %
%%%%%%%%%%%%%%%%%%%%%%%%%%%%%%%%%%%%%%%%%%%%%%%%%%%%%%%%%%%%%%%%%%%%%%%%%%%%%%%

\section{Hund'sche Regeln}
\label 1

Die Schalen werden so besetzt, dass zuerst das s-Orbital, dann das p-Orbital
gefüllt wird, und dort erst die Spins $\uparrow$ und dann $\downarrow$ besetzt
werden. Somit ergeben sich die folgenden Belegungen:

\subsection{Sauerstoff}

Im Sauerstoff, das 8 Valenzelektronen hat, werden die äußeren Schalen wie folgt
besetzt:
\[
	\underset{\text{2s}}{\fbox{$\uparrow \downarrow$}}
	\quad
	\underset{\text{2p}}{\fbox{$\uparrow\downarrow$}\fbox{$\uparrow\downarrow$}\fbox{$\uparrow\phantom\downarrow$}}
\]

Der Gesamtdrehimpuls aus dem p-Orbital sind 4 mal $\ell = 1$, also ist $L = 4$.
Der Spin ist $S = 3 \uparrow + 1 \downarrow = 1$. Somit ist $J = 5$ und $2s + 1
= 3$. Dies ist also $^3\mathrm P_5$ und $\mathrm{[He] 2s^2 \, 2p^4}$.

Wenn allerdings der Gesamtdrehimpuls $L$ die Summe der $m_\ell$ ist, dann ist
$L = 1 + 0 + (-1) + 1 = 1$. Somit wäre der Drehimpuls $J = 2$ und die
spektroskopische Notation $\mathrm{^3P_2}$.

\subsection{Stickstoff}

Besetzung der Schalen:
\[
	\underset{\text{2s}}{\fbox{$\uparrow \downarrow$}}
	\quad
	\underset{\text{2p}}{\fbox{$\uparrow\downarrow$}\fbox{$\uparrow\phantom\downarrow$}\fbox{$\uparrow\phantom\downarrow$}}
\]

Analog zum Sauerstoff, allerdings ist hier noch ein Elektron weniger auf einem
$m_\ell = 0$ Platz. Somit ändert sich nur der Spin auf $S = \sfrac 32$. Ich
erhalte $^4\mathrm P_{\sfrac 32}$ und $\mathrm{[He] 2s^2 \, 2p^3}$.

\subsection{Eifel}

Welches Element soll dies sein?

\fehlt

\subsection{Magnetische Quantenzahl}

$m_J$ gibt die $z$-Komponente des Gesamtdrehimpulses an. Wenn diese entartet
ist, bedeutet dies eine komplette Kugelsymmetrie.

%%%%%%%%%%%%%%%%%%%%%%%%%%%%%%%%%%%%%%%%%%%%%%%%%%%%%%%%%%%%%%%%%%%%%%%%%%%%%%%
%                                    Ende                                     %
%%%%%%%%%%%%%%%%%%%%%%%%%%%%%%%%%%%%%%%%%%%%%%%%%%%%%%%%%%%%%%%%%%%%%%%%%%%%%%%

\IfFileExists{\bibliographyfile}{
	%\bibliography{\bibliographyfile}
}{}

\end{document}

% vim: spell spelllang=de
