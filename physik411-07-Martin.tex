% Copyright © 2013 Martin Ueding <dev@martin-ueding.de>
%
\input{header.tex}

\usepackage{tikz}
\usetikzlibrary{calc}

\newcommand{\themodul}{physik411}
\newcommand{\thegruppe}{Gruppe 2 -- Florian Seidler}
\newcommand{\theuebung}{7}

\ifoot{\footnotesize{Martin Ueding}}
\ihead{\themodul{} -- Übung \theuebung}
\ofoot{\footnotesize{\thegruppe}}

\def\thesubsection{\thesection\alph{subsection}}

\title{\themodul{} -- Übung \theuebung}
\subtitle{\thegruppe}
\author{
	Martin Ueding \footnote{\href{mailto:mu@uni-bonn.de}{mu@uni-bonn.de}}
}

\hypersetup{
	pdftitle={\themodul {} - Übung \theuebung},
}

\begin{document}

\maketitle

\begin{center}
	\ccbysadetitle
\end{center}

\begin{table}[h]
	\centering
	\begin{tabular}{l|c|c|c|c|c}
		Aufgabe
		& \ref 1
		& \ref 2
		& \ref 3
		& \ref 4
		& $\sum$   \\
		\hline
		Punkte
		& \punkte / 9
		& \punkte / 12
		& \punkte / 14
		& \punkte / 6
		& \punkte / 41
	\end{tabular}
\end{table}

%%%%%%%%%%%%%%%%%%%%%%%%%%%%%%%%%%%%%%%%%%%%%%%%%%%%%%%%%%%%%%%%%%%%%%%%%%%%%%%
%                              Hund'sche Regeln                               %
%%%%%%%%%%%%%%%%%%%%%%%%%%%%%%%%%%%%%%%%%%%%%%%%%%%%%%%%%%%%%%%%%%%%%%%%%%%%%%%

\section{Hund'sche Regeln}
\label 1

\subsection{Sauerstoff}

Im Sauerstoff, das 8 Valenzelektronen hat, werden die äußeren Schalen wie folgt
besetzt:
\[
	\underset{\text{2s}}{\fbox{$\uparrow \downarrow$}}
	\quad
	\underset{\text{2p}}{\fbox{$\uparrow\downarrow$}\fbox{$\uparrow\downarrow$}\fbox{$\uparrow\phantom\downarrow$}}
\]

\subsection{Stickstoff}

\[
	\underset{\text{2s}}{\fbox{$\uparrow \downarrow$}}
	\quad
	\underset{\text{2p}}{\fbox{$\uparrow\downarrow$}\fbox{$\uparrow\phantom\downarrow$}\fbox{$\uparrow\phantom\downarrow$}}
\]

\subsection{Eifel}

\subsection{Magnetische Quantenzahl}

$m_J$ gibt die $z$-Komponente des Gesamtdrehimpulses an. Wenn diese entartet
ist, bedeutet dies eine komplette Kugelsymmetrie.

%%%%%%%%%%%%%%%%%%%%%%%%%%%%%%%%%%%%%%%%%%%%%%%%%%%%%%%%%%%%%%%%%%%%%%%%%%%%%%%
%                                    Ende                                     %
%%%%%%%%%%%%%%%%%%%%%%%%%%%%%%%%%%%%%%%%%%%%%%%%%%%%%%%%%%%%%%%%%%%%%%%%%%%%%%%

\IfFileExists{\bibliographyfile}{
	%\bibliography{\bibliographyfile}
}{}

\end{document}

% vim: spell spelllang=de
